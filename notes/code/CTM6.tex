\documentclass [12pt]{article}
\setlength{\oddsidemargin}{0.1in}
\setlength{\evensidemargin}{0.1in}
\setlength{\topmargin}{-.7in}
\setlength{\textheight}{9.25in}
\setlength{\textwidth}{6.5in}
\usepackage{caption}
\usepackage{subcaption}
\usepackage{enumerate}
\usepackage{framed}
\usepackage{epsfig}
\usepackage{changebar}
\usepackage{amsfonts}
\usepackage{amsmath}
\usepackage{amssymb}
\usepackage{graphicx}
\usepackage{amssymb}
\usepackage{stmaryrd}
\graphicspath{ {images/} }
\usepackage{listings}
\usepackage[usenames,dvipsnames]{color}
\usepackage{multicol}
\usepackage{mathtools}
\DeclarePairedDelimiter\ceil{\lceil}{\rceil}
\DeclarePairedDelimiter\floor{\lfloor}{\rfloor}
\setlength{\columnsep}{1cm}
% mathematical commands
\newcommand{\zos}{{\{ 0,1\}^{\ast}}}
\newcommand{\zoi}{{\{ 0,1\}^{\infty}}}
\newcommand{\zon}{{\{ 0,1\}}}
\newcommand{\zov}[1]{{\{ 0,1\}^{#1}}}
\newcommand{\ccc}{{{\cal C}}}
\newcommand{\gggg}{{{\cal G}}}
\newcommand{\nat}{{{\cal N}}}
\newcommand{\rr}{{{\bf RAND}}}
\newcommand{\pref}{{\sqsubset}}
\newcommand{\da}{{\downarrow}}
\newcommand{\ot}{{\otimes}}
\newcommand{\fann}{{\forall n\in \nat}}
\newcommand{\pow}{{{\cal P}}}
\newcommand{\nll}{{{\bf NULL}}}
\newcommand{\nvc}[1]{{{\bf e_{#1}}}}
\newcommand{\st}{{\Sigma_{2}^{A}}}
\newcommand{\ov}[1]{{\overline{#1}}}
\newcommand{\provided}{{\hspace{.1in}:-\hspace{.1in}}}
\begin{document}
\begin{center}\title*{\Large \S \; 6 Gram-Schimdt Process}\\\author*{Jessica Wei} \end{center}
\normalsize
\begin{framed}
\noindent\textbf{DEF} $|$ Inner Product\\
Let $\overrightarrow{x},\overrightarrow{y},\in\mathbb{R}^n$, the inner product of $\overrightarrow{x}$ with $\overrightarrow{y}$
\[<\overrightarrow{x},\overrightarrow{y}>=x_1y_1+x_2y_2+...+x_ny_n=\overrightarrow{x}\cdot\overrightarrow{y}\]
\end{framed}
\begin{framed}
\noindent\textbf{DEF} $|$ Normalized\\
A vector $\overrightarrow{x}\in\mathbb{R}^n$ is normalized if $||\overrightarrow{x}||_2=1.$
\end{framed}
\noindent\textbf{Example. } Find $<\overrightarrow{x},\overrightarrow{x}>$ where $\overrightarrow{x}\in\mathbb{R}^n$. $\overrightarrow{x}=[x_1,x_2,...,x_n]$
\[\overrightarrow{x},\overrightarrow{x}>=x_1\cdot x_1+x_2\cdot x_2+...+x_n\cdot x_n=x_1^2+x_2^2+...+x_n^2\]
\[=||\overrightarrow{x}||_2^2\]
*NOTE: \begin{enumerate}
    \item $||\overrightarrow{x}||_2=\sqrt{<\overrightarrow{x},\overrightarrow{x}>}$
    \item For any vector $\overrightarrow{x}\in\mathbb{R}^n$, we can form the normalized vector $\overrightarrow{y}$ of $\overrightarrow{x}$ by $\overrightarrow{y=\frac{1}{||\overrightarrow{x}||_2}}\overrightarrow{x}$
\end{enumerate}
i.e. any vector of the form $\overrightarrow{y=\frac{1}{||\overrightarrow{x}||_2}}\overrightarrow{x}$ is a normalized vector.\\\\
\textbf{Example.} Normalize $\overrightarrow{x}=[1,1]^T$
\[||\overrightarrow{x}||_2=\sqrt{1^2+1^2}=\sqrt{2}\]
\[\Rightarrow \overrightarrow{y}=\frac{1}{\sqrt{2}\overrightarrow{x}}\]
\[=[\frac{\sqrt{2}}{2},\frac{\sqrt{2}}{2}]^T\]
Verifying:
\[||\overrightarrow{y}||_2=\sqrt{(\frac{\sqrt{2}}{2})^2+(\frac{\sqrt{2}}{2})^2}\]
\[=\sqrt{\frac{2}{4}+\frac{2}{4}}\]
\[=\sqrt{1}=1\]
\pagebreak
\\
\textbf{Properties of Inner Product.} Let $\overrightarrow{x},\overrightarrow{y},\overrightarrow{z}\in\mathbb{R}^n$
\begin{enumerate}
    \item $<\overrightarrow{x},\overrightarrow{x}> > 0$
    \item $<\overrightarrow{x},\overrightarrow{x}> = 0$ if and only if $\overrightarrow{x}=\overrightarrow{0}$
    \item $<>\overrightarrow{x}m,\overrightarrow{y} = <\overrightarrow{y},\overrightarrow{x}>$
    \item $\lambda<\overrightarrow{x},\overrightarrow{y}>=<\lambda\overrightarrow{x},\overrightarrow{y}>$ Proof E.C.
    \item $<\overrightarrow{x}+ \overrightarrow{z},\overrightarrow{y}>=<\overrightarrow{x}+\overrightarrow{y}>+<\overrightarrow{z},\overrightarrow{y}>$ Proof E.C.
    \item $<\overrightarrow{0},\overrightarrow{y}>=0$
\end{enumerate}
\begin{framed}
\noindent\textbf{CLAIM. }Let $\overrightarrow{u},\overrightarrow{v}\in\mathbb{R}^n$
\[<\overrightarrow{u},\overrightarrow{v} = ||\overrightarrow{u}||_2\cdot ||\overrightarrow{v}||_2\cos{\theta}>\]
Where $\theta$ is the angle between the vectors.
\end{framed}
\textbf{Proof.} E.C. (Hint: Consider Law of Cosines)\\
\\
\pagebreak
\\
\textbf{Gram-Schimdt Process}
\begin{framed}
\noindent\textbf{DEF} $|$ Orthogonal\\
Let $\overrightarrow{x},\overrightarrow{y}\in\mathbb{R}^n$. We say $\overrightarrow{x}$ is orthogonal to $\overrightarrow{y}$ if $<\overrightarrow{x},\overrightarrow{y}>=0$ i.e. graphically $\overrightarrow{x} \bot \overrightarrow{y}$. \\
A set of vectors is orthogonal if the vectors are orthogonal to each other.
\end{framed}
\noindent\textbf{Example. }Determine if the set of vectors are orthogonal.\\
\begin{enumerate}
    \item $M={[\frac{1}{\sqrt{2}},[\frac{1}{\sqrt{2}},0]^T,[[\frac{1}{\sqrt{2}},-[\frac{1}{\sqrt{2}},0]^T,[0,0,1]}$\\
    $<\overrightarrow{m_1},\overrightarrow{m_2}>=([\frac{1}{\sqrt{2}})^2-\frac{1}{2}+0=\frac{1}{2}-\frac{1}{2}+0=0$\\
    $<\overrightarrow{m_1},\overrightarrow{m_3}>=0+0+0=0$\\
    $<m_2,m_3>=0+0+0=0$
\end{enumerate}
$\Rightarrow$ The vectors are orthogonal, the set is orthogonal
\begin{framed}
\noindent\textbf{DEF} $|$ Orthonormal\\
A set of vectors in $\mathbb{R}^n$ is orthonormal if the vectors are orthogonal and they are normalized. i.e. they have a magnitude of 1
\end{framed}
\noindent\textbf{Example.} Determine if the set $M$ from the previous example is orthonormal.\\
Ans:
\begin{enumerate}
    \item The vectors are orthogonal
    \item $||\overrightarrow{m}_1||_2 = \sqrt{([\frac{1}{\sqrt{2}})^2+([\frac{1}{\sqrt{2}})^2+0^2}=\sqrt{\frac{1}{2}+\frac{1}{2}+0}= 1 = ||\overrightarrow{m}_2||_2$\\
    $||\overrightarrow{m_3}||_2=\sqrt{0^2+0^2+1^2}=\sqrt{1}=1$\\
\end{enumerate}
Hence, M is orthonormal.
\begin{framed}
\noindent\textbf{DEF} $|$ Kronecker Delta\\
The Kronecker delta, $d_{ij}$, is a relation defined by\\
$d_{ij} = 1$ if $i=j$\\
$d_{ij} = 0$ if otherwise
*Remarks:
\begin{enumerate}
    \item The set ${\overrightarrow{x}_1,\overrightarrow{x}_2,...,\overrightarrow{x}_n}$ is orthonormal if \begin{enumerate}
        \item $<\overrightarrow{x}_i,\overrightarrow{x}_j>= d_{ij}$ for all $i,j\in[1,n]$
        \item $||\overrightarrow{x}_i||=1\indent i=1,2,...,n$
        \item $=<\overrightarrow{x_i},\overrightarrow{x_i}>=||x_i||^2=1^2=1$ when $i-j$
        \item $<\overrightarrow{x_i},\overrightarrow{x_j}>=0$ when $i\neq j$ 
        \item $=d_{ij}$
    \end{enumerate}
    \item A set of orthogonal vectors can be made orthonormal by normalizing each vector ($\frac{1}{||\overrightarrow{x}||}\overrightarrow{x}$)
    \item Any orthogonal/orthonormal set that spans a vector space $V$ is a bases for $V$. Proof: E.C.
\end{enumerate}
\end{framed}
\noindent\textbf{Motivation for Gram-Schimdt Process}
\begin{enumerate}[\quad*]
    \item The Grow Algorithm allows us to find a basis from a set of vectors $T$, for some vector space.
    \item What if we want the basis to be orthonormal?\begin{enumerate}
        \item i.e. $B={\overrightarrow{u},\overrightarrow{x}}$ is a basis for $\mathbb{R}^2$
        \item Notice, $\overrightarrow{x}=\overrightarrow{x}_{\parallel}+\overrightarrow{x}_\bot$ where $\overrightarrow{x}_\parallel = \alpha\overrightarrow{u}$ for $\alpha\in\mathbb{R}$
        \item If we can find $\alpha$, then $\overrightarrow{x}_\bot = \overrightarrow{x}-\overrightarrow{x}_\parallel$ and $\overrightarrow{x}_\bot=\overrightarrow{x}-\alpha\overrightarrow{u}$
        \item In such a case ${\overrightarrow{x}_\bot, \overrightarrow{u}}$ would be an orthogonal basis
        \item and if we normalize the vectors, we obtain an orthonormal basis: $\frac{1}{||\overrightarrow{x}_\bot||}\overrightarrow{x}_\bot,\frac{1}{||\overrightarrow{u}_\bot||}\overrightarrow{u}$
    \end{enumerate}
\end{enumerate}
\noindent\textbf{Goal.} Find $\alpha$.\\
Notice:
\[<\overrightarrow{x}_\bot,\overrightarrow{x}_\parallel>=0\]
\[<\overrightarrow{x}-\overrightarrow{x}_\parallel,\overrightarrow{x}_\parallel>=0\]
\[<\overrightarrow{x}-\alpha\overrightarrow{u},\alpha\overrightarrow{u}>=0\]
\[<\overrightarrow{x}+(-\alpha\overrightarrow{u},\alpha\overrightarrow{u}),\alpha u>=0\]
\[<\overrightarrow{x},\alpha\overrightarrow{u}>+<(-\alpha\overrightarrow{u}),\alpha\overrightarrow{u}>=0\]
\[<\overrightarrow{x},\alpha\overrightarrow{u}>+<(-\alpha\overrightarrow{u}),\alpha\overrightarrow{u}>=0\]
\[<\overrightarrow{x},\alpha\overrightarrow{u}>-\alpha<\overrightarrow{u,\alpha\overrightarrow{u}}>=0\]
\[\alpha<\overrightarrow{x},\overrightarrow{u}>-\alpha^2<\overrightarrow{u},\overrightarrow{u}>=0\]
\[\alpha[<\overrightarrow{x},\overrightarrow{u}>-\alpha<\overrightarrow{u},\overrightarrow{u}>]=0\]
\textbf{Case:} If $\alpha=0$\\
\[\overrightarrow{x}=\overrightarrow{x}_\parallel+\overrightarrow{x}_\bot=\alpha\overrightarrow{u}+\overrightarrow{x}_\bot\]
\[\overrightarrow{x}=\overrightarrow{x}_\bot\]
$\overrightarrow{x}$ is already to $\overrightarrow{u}$\\
\textbf{Case:} $<\overrightarrow{x},\overrightarrow{u}>-\alpha<\overrightarrow{u},\overrightarrow{u}>=0$
\[\alpha<\overrightarrow{u},\overrightarrow{u}>=<\overrightarrow{x},\overrightarrow{u}>\]
\[\alpha=\frac{<\overrightarrow{x},\overrightarrow{u}>}{<\overrightarrow{u},\overrightarrow{u}>}\]
Hence, given $\overrightarrow{x},\overrightarrow{u}$
\[\overrightarrow{x}=\overrightarrow{x}_\parallel+\overrightarrow{x}_\bot\]
\[\overrightarrow{x}_\parallel=\alpha\overrightarrow{u} = \frac{<\overrightarrow{x},\overrightarrow{u}>}{<\overrightarrow{u},\overrightarrow{u}>} \overrightarrow{u}=proj_{\overrightarrow{u}\overrightarrow{x}}\]
$\frac{<\overrightarrow{x},\overrightarrow{u}>}{<\overrightarrow{u},\overrightarrow{u}>} \overrightarrow{u}$ is also known as the projection of $\overrightarrow{x}$ onto $\overrightarrow{u}$\\\\
\textbf{Gram Schimdt Process in $\mathbb{R}^2$}
\begin{enumerate}[\quad*]
    \item INPUT: Basis $B={\overrightarrow{b}_1,\overrightarrow{b_2}}$
    \item OUTPUT: orthonormal basis $N={\overrightarrow{v}_1,\overrightarrow{v}_2}$
    \item Normalize the first vector\\
    $\overrightarrow{v}_1=\frac{1}{||\overrightarrow{b_1}||}\overrightarrow{b_1}$
    \item Form the vector\\
    $\overrightarrow{y}=\overrightarrow{b_1}-proj_{\overrightarrow{b_1}}\overrightarrow{b_2}\Rightarrow \overrightarrow{v_2}=\frac{1}{||\overrightarrow{y}||}\overrightarrow{y}$
\end{enumerate}
\noindent\textbf{Example.} Find an orthonormal basis given the basis for $\mathbb{R}^2$ where $B=\{[2,7],[-3,4]\}$.
\[||\overrightarrow{b_1}||=\sqrt{-3^2+4^2}=\sqrt{9+19}=5\]
\[=\frac{1}{||\overrightarrow{b_1}||}\overrightarrow{b_1}=\frac{1}{5}[-3,4]=[\frac{3}{5}, \frac{4}{5}]\]
\[*\overrightarrow{y}=\overrightarrow{b_1}-proj_{\overrightarrow{b_2}}\overrightarrow{b_1}\]
\[proj_{\overrightarrow{b_2}}\overrightarrow{b_1}=\frac{<\overrightarrow{b_1},\overrightarrow{b_2}>}{<\overrightarrow{b_2},\overrightarrow{b2}>}\overrightarrow{b_2}=\frac{-6+28}{9+16}\overrightarrow{b_2}\]
\[=\frac{22}{25}[3,2]=[\frac{-66}{25},\frac{88}{25}]\]
\[\Rightarrow\overrightarrow{y}=[2,7]-[\frac{-66}{25},\frac{88}{25}]=[\frac{116}{25},\frac{87}{25}]\]
\[\overrightarrow{v_2}=\frac{1}{||\overrightarrow{y}||}\overrightarrow{y}=\frac{1}{||\overrightarrow{y}||}[\frac{116}{25},\frac{87}{25}]\]
$\Rightarrow$ Orthonomral basis $N=\{\overrightarrow{v_1},\overrightarrow{v_2}\}= \{[\frac{3}{5}, \frac{4}{5}],\frac{1}{||\overrightarrow{y}||}[\frac{116}{25},\frac{87}{25}]\}$
\end{document}