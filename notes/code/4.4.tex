\documentclass [12pt]{article}
\setlength{\oddsidemargin}{0.1in}
\setlength{\evensidemargin}{0.1in}
\setlength{\topmargin}{-.7in}
\setlength{\textheight}{9.25in}
\setlength{\textwidth}{6.5in}
\usepackage{caption}
\usepackage{subcaption}
\usepackage{enumerate}
\usepackage{framed}
\usepackage{epsfig}
\usepackage{changebar}
\usepackage{amsfonts}
\usepackage{amsmath}
\usepackage{amssymb}
\usepackage{graphicx}
\usepackage{amssymb}
\usepackage{stmaryrd}
\graphicspath{ {images/} }
\usepackage{listings}
\usepackage[usenames,dvipsnames]{color}
\usepackage{multicol}
\usepackage{mathtools}
\DeclarePairedDelimiter\ceil{\lceil}{\rceil}
\DeclarePairedDelimiter\floor{\lfloor}{\rfloor}
\setlength{\columnsep}{1cm}
% mathematical commands
\newcommand{\zos}{{\{ 0,1\}^{\ast}}}
\newcommand{\zoi}{{\{ 0,1\}^{\infty}}}
\newcommand{\zon}{{\{ 0,1\}}}
\newcommand{\zov}[1]{{\{ 0,1\}^{#1}}}
\newcommand{\ccc}{{{\cal C}}}
\newcommand{\gggg}{{{\cal G}}}
\newcommand{\nat}{{{\cal N}}}
\newcommand{\rr}{{{\bf RAND}}}
\newcommand{\pref}{{\sqsubset}}
\newcommand{\da}{{\downarrow}}
\newcommand{\ot}{{\otimes}}
\newcommand{\fann}{{\forall n\in \nat}}
\newcommand{\pow}{{{\cal P}}}
\newcommand{\nll}{{{\bf NULL}}}
\newcommand{\nvc}[1]{{{\bf e_{#1}}}}
\newcommand{\st}{{\Sigma_{2}^{A}}}
\newcommand{\ov}[1]{{\overline{#1}}}
\newcommand{\provided}{{\hspace{.1in}:-\hspace{.1in}}}
\begin{document}
\begin{center}\title*{\Large \S \; 4.4 Solving Congruences}\\\author*{Jessica Wei} \end{center}
\normalsize
\noindent
%-------------Linear Congruence
\textbf{\subsection*{Linear Congruence}}
\begin{framed}
\noindent\textbf{DEF} $|$ Linear Congruence\\
A congruence of the form 
\[ax=b(\mod m)\]
where $m$ is a positive integer, $a$ and $b$ are integers, and $x$ is a variable, is called a linear congruence.\\
\end{framed}
%------------------GOAL 
\noindent\textbf{Goal.} Solve for $x$\\
\indent Case 1: $a|b$\\
\indent \indent Q: if $a|b$, is the answer $x=\frac{b}{a}\mod m$?\\
\indent\indent Short Answer: Not always, i.e. $2\cdot 7\equiv8\mod6\neq7\equiv4\mod6$\\
\indent\indent Long Answer...... (see below)
\begin{framed}
%------------------LEMMA 4.4.1
\noindent$|$ Lemma 4.4.1 (4.3.2 in textbook)\\
Let $a,b,c\in\mathbb{Z^+}$. If gcd($a$, $b$) = 1, and $a|b\cdot c$, then $a|c$ 
\end{framed}
\noindent\textbf{PROOF}\\
\indent If gcd($a$, $b$) = 1, then $\exists s,t,\in\mathbb{Z}$ such that $a\cdot s\cdot c + t\cdot b\cdot c = c$.\\
\indent Since $a|b\cot c$, $b\cdot c = a\cdot k $ for some $k\in\mathbb{Z}$.\\
\indent Hence, $a\cdot s\cdot c+a\cdot k \cdot t = c \Rightarrow a(s\cdot c + k\cdot t) = c$ where $s\cdot c + k\cdot t \in\mathbb{Z}$\\
\indent$\therefore a|c$
\vspace{0.2in}
%------------------THM 4.3.6
\begin{framed}
\noindent $|$ THM 4.3.6\\
Let $m\in\mathbb{Z^+}$ and $a,b,c\in\mathbb{Z}$. If (1) $a\cdot c \equiv b\cdot a\mod m$ and (2) gcd($a$, $m$) = 1, then $c\equiv b\mod m$
\end{framed}
\noindent\textbf{PROOF}
\begin{enumerate}[\indent]
    \item By (1), $a\cdot c-b\cdot a = m\cdot k \Rightarrow a(c-b) = m\cdot k$. Since gcd($a$, $m$) = 1 and $a|m\cdot k$, then by Lemma 4.4.1, $a|k$.
    \item\quad$\Rightarrow c-b=m(\frac{k}{a}$
    \item\quad $c-b = m\cdot q$ when $q = \frac{k}{a}\in\mathbb{Z}$
    \item\quad $c\equiv b\mod m$
    \item Hence, (1) $a\cdot x \equiv b\mod m$ will have solution $x\equiv \frac{b}{a}\mod m$ if (2) gcd($a$, $m$) = 1.
\end{enumerate}
\pagebreak
%------------------EX 1
\noindent\textbf{Example 1.} Solve $6\cdot x\equiv12\mod7$\\
\indent Since gcd(6, 7) = 1, then $x\equiv2\mod7$\\
\indent\textbf{Answer:} $x\equiv2\mod7$
\vspace{0.2in}\\
%------------------GOAL CON'T
\noindent\textbf{Goal. (con't)} Solve for $x$\\
\indent Case 2: $a\nmid b$\\
\[a\cdot x \equiv b\mod m\]
\indent Idea: If we can find $\Bar{a}\in\mathbb{Z}$ such that $\Bar{a}a\equiv1\mod m$ then $\Bar{a}ax\equiv x\mod m$.
\[x\equiv \Bar{a}bc\mod m\]
%------------------THM 4.3.6
\begin{framed}
\noindent$|$ THM 4.4.1\\
If gcd($a$, $m$) = 1, $m>1$, then there exists $\Bar{a}\in\mathbb{Z}$ unique modulo m such that $\Bar{a}a\equiv1\mod m$.
\end{framed}
\noindent\textbf{PROOF}
\begin{enumerate}[\indent]
    \item Since gcd($a$, $m$) = 1, then $\exists s,t\in\mathbb{Z}$
    \item$as+mt=1$
    \item\quad $\Rightarrow (as+nt) - 1 = 0m$
    \item\quad $\Rightarrow as + mt \equiv 1\mod m$
    \item\quad $\Rightarrow as + 0 \equiv 1\mod m$
    \item\quad $\Rightarrow as\equiv 1\mod m$ because $m|mt$
    \item so $\Bar{a}$ is actually the Bezout Coefficient of $a$ with $m$.
\end{enumerate}
\vspace{0.2in}
\noindent\textbf{Uniqueness} Assume $\exists u\in\mathbb{Z} such that$
\begin{enumerate}[\quad]
    \item\quad $ua\equiv 1\mod m$
    \item\quad $ua\Bar{a}\equiv\Bar{a}\mod m$ \quad ($a\Bar{a}\equiv 1$)
    \item\quad $u\equiv\Bar{a}\mod m$
    \item So $\Bar{a}$ is unique modulo m
\end{enumerate}
\vspace{0.2in}
\noindent\textbf{Summarize} 
\begin{enumerate}[\quad]
    \item$ax\equiv b\mod m$
    \item Requirement: gcd($a$,$m$) = 1
    \item Case 1: If $a|b$, then $x=\frac{b}{a}\mod m$.
    \item Case 2: If $a\nmid b$, then $x=\Bar{a}b\mod m$ where $\Bar{a}$ is the Bezout Coefficient(a.k.a. "inverse modulo m") of gcd($a$, $m$).
\end{enumerate}
\pagebreak
%------------------EX 2
\noindent\textbf{Example 2.} Find the inverse if possible.
\begin{enumerate}[\quad(a)]
    \item 3 modulo 7 = -2 $\equiv$ 5 \quad ($-2\mod7=5$)\\
    \quad $7=2\cdot3+1\Rightarrow1=7-2\cdot3$ where 2 is the Bezout Coefficient of 3\\
    \quad $3 = 3\cdot1+0$ \\
    \quad\textbf{Answer:} 5
    \item 101 modulo 4620\\
    \quad $4620=45\cdot101+75 \Rightarrow 75=4620-45\cdot101$\\
    \quad $101 = 1\cdot75+26\Rightarrow26=101-75$\\
    \quad $75=2\cdot26+23\Rightarrow23=75-2\cdot26$\\
    \quad $26=1\cdot23+3\Rightarrow3=26-1\cdot23$\\
    \quad $23=7\cdot3+2\Rightarrow2=23-7\cdot3$\\
    \quad $3=1\cdot2+1$\quad$\checkmark$\quad$\Rightarrow1=3-2$\\
    \quad $2=2\cdot1+0$\\
    ------------------------------------\\
    \quad $1=3-2 \Rightarrow 3-(23-7\cdot3)$\\
    \quad $=(8\cdot3)-23\Rightarrow(8(26-23))-23$\\
    \quad $=8\cdot26-9\cdot23\Rightarrow8\cdot26-9(75-2\cdot26)$\\
    \quad $=26\cdot26-9\cdot75\Rightarrow26(101-75)-9\cdot75$\\
    \quad $=26\cdot101-35\cdot78\Rightarrow26\cdot101-35(4620-48\cdot101)$\\
    \quad $=26\cdot101-35(4620)+1575\cdot101\Rightarrow1601\cdot101-35\cdot4620$\\
    \quad \textbf{Answer:} $\Bar{a} = 1601$
\end{enumerate}
\vspace{0.2in}
%------------------EX 3
\noindent\textbf{Example 3.} Solve $3x\equiv4\mod7$
\begin{enumerate}[\quad]
    \item * gcd(3, 7) = 1 \quad\checkmark
    \item * inverse of 3 modulo 7 = 5
    \item $\Rightarrow5\cdot 3x\equiv5\cdot4\mod7$
    \item $\Rightarrow x \equiv 20\mod 7$
    \item $x\equiv6\mod7$
    \item $x-6=7k$ where $k\in\mathbb{Z}$
    \item \textbf{Answer:} $x=7k+6$
\end{enumerate}
Check
\begin{enumerate}[\quad]
    \item $k=0$\quad $x=6$ \quad $3\cdot6\stackrel{?}{=}4\mod7$ \quad \checkmark
    \item $k=1$\quad $x=13$ \quad $3\cdot13\stackrel{?}{=}4\mod7$ \quad \checkmark
\end{enumerate}
\pagebreak
%------------------GOAL
\noindent\textbf{Goal.} Solve a system of linear congruences.
\begin{enumerate}[\quad]
    \item $x\equiv a_1\mod m_1$
    \item $x\equiv a_2\mod m_2$
    \item $x\equiv a_3\mod m_3$
\end{enumerate}
OR more compactly...\\
\indent $x\equiv a_i\mod m_i$ where $i=1,2,3....n$
\begin{framed}
\noindent$|$ Chinese Remainder THM\\
The system
\[x\equiv a_i\mod m \quad\quad i=1,...,n\]
has a unique solution modulo $m = m_1\cdot m_2\cdot ... m_n$ provided that:
\begin{enumerate}[\quad (i)]
    \item gcd($m_i$, $m_j$) = 1\quad $i\neq j$
    \item $m_i > 1$
\end{enumerate}
\end{framed}
\noindent\textbf{PROOF}
\begin{enumerate}[\quad]
    \item Existence: For $m$ $M_k=\frac{m}{m_k} = m_1\cdot m_2 ... m_{k-1}\cdot m_{k+1} ... m_n$ \quad (Notice that $m_k$ is missing)
    \item Then gcd($M_k$, $m_k$) = 1\quad i.e. they will be relatively prime.
    \item By THM 4.4.1, $\exists\Bar{M}_k\in\mathbb{Z}$ such that $\Bar{M}_kM_k\equiv1\mod M_k$. We claim that $x=a_1M_1\Bar{M}_1+a_2M_2\Bar{M}_2+...a_nM_n\Bar{M}_n$ solves the system.
    \item To show that this is true, notice that for any $j=1,... n$: $M_j\equiv0\mod M_k$ for $j\neq k$
    \item This is because $M_j=m_1m_2...m_{j-1}m_{j+1}...m_km_{k+1}...m_n$
    \item So $m_k|M_j$. Then... (anything $M_k$ is going away [= 0])
    \item $x=a_1M_1\Bar{M}_1+...+a_1M_k\Bar{M}_k+...a_nM_n\Bar{M}_n\equiv 0+0+...+a_1M_k\Bar{M}_k+0...+0\mod m_k$
    \item $\Rightarrow x=a_kM_k\Bar{M}_k\mod M_k$
    \item $\Rightarrow x=a_x\cdot1\mod m_k$
    \item $\Rightarrow x=a_x\mod m_k$
    \item This is true for any $k=1,2,.... n$
    \item Hence $x=a_1M_1\Bar{M}_1+a_2M_2\Bar{M}_2+...a_nM_n\Bar{M}_n$ solves the system.
\end{enumerate}
\pagebreak
%------------------EX 1
\noindent\textbf{Example 1.} Solve
\[x\equiv2\mod3\]
\[x\equiv3\mod5\]
\[x\equiv2\mod7\]
with the Chinese Remainder Theorem
\begin{enumerate}[\indent]
    \item *$m_1=3,m_2=5,m_3=7,m=3\cdot5\cdot7=105$
    \item *$M_1=5\cdot7=35,M_2=3\cdot7=21,M_3=3\cdot5=15$
    \item *Need $\Bar{M}_1\cdot35\equiv1\mod3\Rightarrow2$, $\Bar{M}_2\cdot21\equiv1\mod5\Rightarrow1$, $\Bar{M}_3\cdot15\equiv1\mod7\Rightarrow1$
    \item $x=a_1M_1\Bar{M}_1+a_2M_2\Bar{M}_2+a_3M_3\Bar{M}_3$
    \item\quad=$2\cdot35\cdot2+3\cdot21\cdot1+2\cdot15\cdot1 = 140+63+30=233\equiv?\mod105$ where $? = 23$
    \item$\Rightarrow x\equiv23\mod105\Rightarrow x-23=105k$\quad $k\in\mathbb{Z}$
    \item\textbf{Answer:} $x=105k+23$
\end{enumerate}
\vspace{0.2in}
%------------------EX 2
\noindent\textbf{Example 2.} Solve
\[x\equiv2\mod3\quad(1)\]
\[x\equiv3\mod5\quad(2)\]
\[x\equiv2\mod7\quad(3)\]
with substitution
\begin{enumerate}[\indent]
    \item $(1)\quad x-2=3k\quad k\in\mathbb{Z}\Rightarrow x=3k+2\quad(A)$
    \item Plug (A) into (2).
    \item\quad $3k+2\equiv3\mod5\Rightarrow mk\equiv1\mod5$ \quad\quad $3\Bar{a}\equiv1\mod5$ where $\Bar{a}\equiv2$
    \item\quad $2\cdot3k\equiv2\cdot1\mod5\Rightarrow k\equiv2\mod5$
    \item\quad $k-2=5q\quad q\in\mathbb{Z}\Rightarrow k = 5q+2\quad(B)$
    \item Plug (B) into (A) -- This results in an $x$ that solves (1) and (2).
    \item\quad $x=3k+2 = 3(5q+2)+2\Rightarrow x=15q+8\quad(C)$
    \item Plug (C) into (3).
    \item\quad $15q+8\equiv2\mod7$
    \item\quad $15q\equiv-2\mod7$
    \item\quad $15q\equiv1\mod7$ where $\Bar{a}15\equiv 1\mod7$, $\Bar{a}\equiv1$
    \item\quad $1\cdot15q\equiv1\cdot1\mod7$
    \item\quad $q=1\mod7\Rightarrow q=7u+1\quad(D)$
    \item Plug (D) into (C) to obtain an $x$ that satisfies the whole system.
    \item\quad $x=15(7u+1)+8 = 105u+15+8$
    \item\textbf{Answer:} $x=105u+23$
\end{enumerate}

%-------------Fermat's Little THM & Modular Exponentiation
\textbf{\subsection*{Fermat's Little Theorem \& Modular Exponentiation}}
\begin{framed}
\noindent$|$ THM 4.4.2\\
If $p$ is prime \& $p\nmid a$ then...
\begin{enumerate}[(i)]
    \item $a^{p-1}\equiv1\mod p$
    \item $a^p\equiv a\mod p$
\end{enumerate}
\end{framed}
%------------------EX 3
\noindent\textbf{Example 3.} Find each of the following
\begin{enumerate}[a)]
    \item $7^{222}\mod11=?\mod m$\\
    \quad *Idea: Recall $a\mod m = b\mod m$ if and only if $a\equiv b\mod m$. We will reduce $7^{222}$ into a smaller number $b$ modulo $m$. i.e. $7^{222}\equiv ? smaller \mod 11$\\
    \quad Notice 11 is prime \& $p=11\nmid7$\\
    \quad Hence, by FLT...
    \[7^{11-1}\equiv1\mod11\Rightarrow7^{10}\equiv1\mod11\]
    *Claim: If $a=b\mod m$. then $a^k=b^k\mod m$ for $k\in\mathbb{Z^+}$\\
    \quad *Proof: $(7^{10})^{21}\equiv 1^{20}\mod11$\\
    \quad $7^{222} = 1\mod11\Rightarrow7^2\cdot7^{200}=7^2\cdot1\mod11$\\
    \quad $7^{222}\equiv7^2\mod11\Rightarrow7^{222}\mod11\equiv7^2\mod11=49\mod11$\\
    \quad\textbf{Answer:} $5\mod11$
    \item $7^{121}\mod13 = ?\mod13$\\
    *Goal: $7^{121}=?\mod13$, $p=13$, $13\nmid7$\\
    \quad$\Rightarrow7^{13-1}\equiv1\mod13$ by FLT\\
    $\Rightarrow(7^{12})^{10}\equiv1^{10}\mod13$\\
    $7\cdot7^{120}\equiv1\cdot7\mod13$\\
    $7^{121} = 7\mod13$\\
    $\Rightarrow 7^{121}\mod13 = 7\mod13$\\
    \textbf{Answer:} 7
\end{enumerate}
\pagebreak
\end{document}