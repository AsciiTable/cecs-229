\documentclass [12pt]{article}
\setlength{\oddsidemargin}{0.1in}
\setlength{\evensidemargin}{0.1in}
\setlength{\topmargin}{-.7in}
\setlength{\textheight}{9.25in}
\setlength{\textwidth}{6.5in}
\usepackage{caption}
\usepackage{subcaption}
\usepackage{enumerate}
\usepackage{framed}
\usepackage{epsfig}
\usepackage{changebar}
\usepackage{amsfonts}
\usepackage{amsmath}
\usepackage{amssymb}
\usepackage{graphicx}
\usepackage{amssymb}
\usepackage{stmaryrd}
\graphicspath{ {images/} }
\usepackage{listings}
\usepackage[usenames,dvipsnames]{color}
\usepackage{multicol}
\usepackage{mathtools}
\DeclarePairedDelimiter\ceil{\lceil}{\rceil}
\DeclarePairedDelimiter\floor{\lfloor}{\rfloor}
\setlength{\columnsep}{1cm}
% mathematical commands
\newcommand{\zos}{{\{ 0,1\}^{\ast}}}
\newcommand{\zoi}{{\{ 0,1\}^{\infty}}}
\newcommand{\zon}{{\{ 0,1\}}}
\newcommand{\zov}[1]{{\{ 0,1\}^{#1}}}
\newcommand{\ccc}{{{\cal C}}}
\newcommand{\gggg}{{{\cal G}}}
\newcommand{\nat}{{{\cal N}}}
\newcommand{\rr}{{{\bf RAND}}}
\newcommand{\pref}{{\sqsubset}}
\newcommand{\da}{{\downarrow}}
\newcommand{\ot}{{\otimes}}
\newcommand{\fann}{{\forall n\in \nat}}
\newcommand{\pow}{{{\cal P}}}
\newcommand{\nll}{{{\bf NULL}}}
\newcommand{\nvc}[1]{{{\bf e_{#1}}}}
\newcommand{\st}{{\Sigma_{2}^{A}}}
\newcommand{\ov}[1]{{\overline{#1}}}
\newcommand{\provided}{{\hspace{.1in}:-\hspace{.1in}}}
\begin{document}
\begin{center}\title*{\Large \S \; 4.3 Primes \& GCD}\\\author*{Jessica Wei} \end{center}
\normalsize
\noindent
%-------------PRIME-------------------
\textbf{\subsection*{Prime}}
\begin{framed}
\noindent\textbf{DEF} $|$ Prime\\
Let $p\in\mathbb{Z}^+$. We say $p$ is a prime if it is only divisible by $p$ (itself) and 1. i.e. 2, 3, 5, 7...\\
Note: If an integer is not prime, then it is called composite.
\end{framed}
%--------------------THM 4.3.1
\begin{framed}
\noindent\textbf{THM 4.3.1} $|$ \\
Every composite integer has a prime factorization.
\[a = p_1^{a_1}\cdot p_2^{a_2}\cdot ... p_k^{a_k}\]
\end{framed}
%------------------EX 1
\noindent\textbf{Example 1.} Find the prime factorization of 100.\\
\[100 = 10 \cdot 10\]
\[10 = 2 \cdot 5\]
\[\therefore 100 = 2\cdot 2 \cdot 5 \cdot 5\]
\[\textbf{Answer: }100 = 2^2\cdot5^2\]
%-----------------THM 4.3.2
\begin{framed}
\noindent\textbf{THM 4.3.2} $|$ \\
If $n$ is composite, then $\exists p\in\mathbb{Z}$such that $p$ is prime, $p\leq\sqrt{n}$, and $p|n$.\\\\
\textbf{PROOF:} Since n is composite, there exists $a,b\in\mathbb{Z}$ such that $n=a\cdot b$ and $a\neq n \neq b$.\\\\
Case 1: $a > \sqrt{n}$ and $b>\sqrt{n}$
\begin{enumerate}[\quad]
    \item Then $a\cdot b > \sqrt{n}$, $\sqrt{n} = n$
    \item $\therefore a \cdot b > n$
    \item This contradicts $a\cdot b = n$
    \item Hence, it must be true that either $a \leq \sqrt{n}$ or $b \leq \sqrt{n}$
\end{enumerate}
Assume $a\leq \sqrt{n}$. If this is not true, the following argument can be made for b.
\begin{enumerate}[\quad]
    \item Possibility \#1. $a$ is prime. Then $a\leq n$ satisfies and $n=a\cdot b\Rightarrow a|n$ satisfies.
    \item Possibility \#2. $a$ is composite. Then by THM 4.3.1, $a$ has a prime factorization so that $n=a\cdot b = p_1^{a_1}....p_x^{a_x}\cdot b$. SO any prime $p_i$ in this factorization divides n $(p_i|n)$ and $ p_i^{a_i}....p_x^{a_x} = a \leq \sqrt{n}$ which implies $p_i\leq \sqrt{n}$
\end{enumerate}
\end{framed}

%------------------EX 2
\noindent\textbf{Example 2.} Show that 61 is prime.\\
\vspace{0.1in}
\quad Assume that 61 is composite. In such a case $\exists p \leq \sqrt{61}$ such that $p|61$.
\begin{enumerate}[\quad]
    \item $\sqrt{61}\approx7...$, $p\leq\sqrt{61}\approx7...$
    \item Range of primes: 2, 3, 5, 7...
    \item $2\nmid61$, $3\nmid61$, $5\nmid61$, $7\nmid61$
    \item $\Rightarrow$ Contradiction hence 61 cannot be composite. 61 must be prime.
\end{enumerate}
%-----------------THM 4.3.3
\begin{framed}
\noindent\textbf{THM 4.3.3} $|$ \\
There are infinitely many primes.\\\\
\textbf{PROOF:} Assume that there are only k-number primes $p_1,p_2.... p_k$. Form the number\\
\[Q=p_1...p_x+1\]\\
Assuming Q is composite, $Q$ must have a prime factorization. Suppose $p$ is a prime factor of Q ($p|Q$).\\
\vspace{0.1cm}
\\
Notice then that $p|Q-p_1....p_k\Rightarrow p|1$, which is impossible. Hence, Q cannot be composite.\\
\vspace{0.1cm}
\\
So Q is prime which is a contradiction to finite number $k$ primes. There are indefinitely-many primes. 
\[2016: 2^{74207281}-1\]
\[2017: 2^{7712321917}-1\]
\end{framed}

%-------------GCD's & LCM's-------------------
\textbf{\subsection*{GCD's \& LCM's}}
%-------------------GCD DEF
\begin{framed}
\noindent\textbf{DEF} $|$ GCD\\
Let $a,b\in\mathbb{Z^+}$, the largest integer $d\in\mathbb{Z^+}$ that divides $a$ and $b$ is the greatest common divider. 
\[d|a \wedge d|b\]
\end{framed}
%-------------------Pairwise Relatively Prime DEF
\begin{framed}
\noindent\textbf{DEF} $|$ Pairwise Relatively Prime\\
Two or more integers are called pairwise relatively prime if the GCD between any two such integers is 1.
\[a_1,a_2...a_k\]
\[gcd(a_i,a_j) = 1\]
\[1<i,j<k\]
\[gcd (4,3) = 1\]
\[i\neq j\]
\end{framed}
%------------------EX 3
\noindent\textbf{Example 3.} What is...
\begin{multicols}{2}
\begin{enumerate}[a)]
    \item gcd(24, 36) = 12\\
    \textbf{Answer: }$12|24$ \& $12|36$
    \item gcd(17, 22) = 1\\
    \textbf{Answer: } Relatively Prime
\end{enumerate}
\end{multicols}
%------------------EX 4
\noindent\textbf{Example 4.}Determine if the integers in the list are pairwise relatively prime.
\begin{enumerate}[a)]
    \item 10, 17, 21\\
    gcd(10, 19) = 1, gcd(10, 24) = 1, gcd(17, 21) = 1\\
    \textbf{Answer: }$\therefore$ pairwise relatively prime
    \item 10, 19, 24\\
    gcd(10, 19) = 1, gcd(10, 24) = \textbf{2}, gcd (19, 24)\\
    \textbf{Answer: }$\therefore$ not pairwise relatively prime
\end{enumerate}
%------------------NOTICE
\textbf{NOTICE: }In general, if we are trying to find the GCD of any two numbers and we consider their prime factorization, then..
\[a=p_1^{a_1}...p_k^{a_k}, b=p_1^{b_1}...p_k^{b_k}\]
\[24=2^3\cdot3, 10=2\cdot 5\]
\[gcd(24,10)=2\]
\[24 = 2^3\cdot3^1, 36 = 2^2\cdot3^2\]
\[gcd = 2^2\cdot 3^1 = 12\]
\[** gcd (a,b) = p_1^{min(a_1b_1}....p_k^{min(a_kb_k}**\]
\vspace{0.2in}
%-------------------LCM DEF
\begin{framed}
\noindent\textbf{DEF} $|$ LCM\\
The Least Common Multiple of $a,b\in\mathbb{Z^+}$ is the smallest integer $m$ such that $a|m$ and $b|m$.
\begin{enumerate}[\quad]
    \item  i.e. lcm(8, 10) = 40
    \item 8: 8, 16, 24, 32, \textbf{40}, 48...
    \item 10: 10, 20, 30, \textbf{40}, 50...
    \item 8 = $2^3$, 10 = $2^3\cdot5^1$
\end{enumerate}
The powers of the prime factors of the least common multiple have to be the highest power present in the prime factors for $a\& b$.
\[LCM (a,b) = p_1^{max(a_1,b_1}...p_k^{max(a_k,b_k}\]
\textbf{NOTE: } Finding the GCD by trail \& error of prime factorization is very costly and slow when trying to program it.
\end{framed}

%-------------Euclidean Algorithm-------------------
\textbf{\subsection*{Euclidean Algorithm}}
%-----------------THM 4.3.4
\begin{framed}
\noindent\textbf{THM 4.3.4} $|$ \\
Let $a,b\in\mathbb{Z^+}$ such that $a=b\cdot q+r$, $r>0$. Then the $gcd(a, b) = gcd(b,r)$\\\\
\textbf{PROOF:} Idea - show that all divisors of $a$ \& $b$ are divisors of b and r because this includes the GCD.\\
i.e. Show $d|a \wedge d|b$
\begin{enumerate}[(i)]
    \item $d|a\Rightarrow a = d\cdot s + a_0$
    \item $b|d\Rightarrow b = d\cdot k+b_0$
\end{enumerate}
To show $d|r$: $a=b\cdot q+r$
\begin{enumerate}[\quad]
    \item $d_s = d\cdot k\cdot q + r$
    \item $\Rightarrow r = d\cdot s-d\cdot k\cdot q$
    \item $\Rightarrow r = d\cdot s - d\cdot k\cdot q$
    \item $\Rightarrow r = d(s-k\cdot q)$ where $s-k\cdot q \in\mathbb{Z}$
    \item $\Rightarrow d|r$
    \item $\Rightarrow d|b$ and $d|r$
\end{enumerate}
Now assume $d|b$ and $d|r$ so that $b=d\cdot k$ \& $r = d\cdot s$
\begin{enumerate}[\quad]
    \item Since $a = b\cdot q+r$
    \item $a=d\cdot k\cdot q + d\cdot s = d(k\cdot q + s)\Rightarrow d|a$ where $k\cdot q+s \in\mathbb{Z}$
    \item $\therefore d|a$ \& $d|b$
\end{enumerate}
Hence all divisors of a \& b are divisors of b \& r including GCD.
\end{framed}
\pagebreak
%------------------EX 5
\noindent\textbf{Example 5. }Find the GCD(a, b) where $a=b\cdot q + r$, then we can successfully reduce the problem of finding GCD(a, b) by dividing the large number by the smaller number, then the smaller number by the remainder until $r = 0$.
\begin{enumerate}[a)]
    \item Find gcd(20, 500)\\
    $500 = 4\cdot 120 + 20$ $\Rightarrow$ gcd(120, 20)\\
    $120 = 6\cdot 20 + 0$ $\Rightarrow$ 20\\
    \textbf{Answer: }gcd(20, 500) = 20
    \item Find gcd(414,662)\\
    $662 = 1\cdot 414 + 248$ $\Rightarrow$ gcd(414, 248)\\
    $414 = 1\cdot 248 + 166$ $\Rightarrow$ gcd(248, 166)\\
    $248 = 1\cdot 166 + 82$ $\Rightarrow$ gcd(166, 82)\\
    $166 = 2\cdot 82 + 2$ $\Rightarrow$ gcd(82, 2)\\
    $82 = 41\cdot 2 + 0$ $\Rightarrow$ 2\\
    \textbf{Answer: }gcd(414, 662) = 2
\end{enumerate}
\end{document}