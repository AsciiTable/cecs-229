\documentclass [12pt]{article}
\setlength{\oddsidemargin}{0.1in}
\setlength{\evensidemargin}{0.1in}
\setlength{\topmargin}{-.7in}
\setlength{\textheight}{9.25in}
\setlength{\textwidth}{6.5in}
\usepackage{caption}
\usepackage{subcaption}
\usepackage{enumerate}
\usepackage{framed}
\usepackage{epsfig}
\usepackage{changebar}
\usepackage{amsfonts}
\usepackage{amsmath}
\usepackage{amssymb}
\usepackage{graphicx}
\usepackage{amssymb}
\usepackage{stmaryrd}
\graphicspath{ {images/} }
\usepackage{listings}
\usepackage[usenames,dvipsnames]{color}
\usepackage{multicol}
\usepackage{mathtools}
\DeclarePairedDelimiter\ceil{\lceil}{\rceil}
\DeclarePairedDelimiter\floor{\lfloor}{\rfloor}
\setlength{\columnsep}{1cm}
% mathematical commands
\newcommand{\zos}{{\{ 0,1\}^{\ast}}}
\newcommand{\zoi}{{\{ 0,1\}^{\infty}}}
\newcommand{\zon}{{\{ 0,1\}}}
\newcommand{\zov}[1]{{\{ 0,1\}^{#1}}}
\newcommand{\ccc}{{{\cal C}}}
\newcommand{\gggg}{{{\cal G}}}
\newcommand{\nat}{{{\cal N}}}
\newcommand{\rr}{{{\bf RAND}}}
\newcommand{\pref}{{\sqsubset}}
\newcommand{\da}{{\downarrow}}
\newcommand{\ot}{{\otimes}}
\newcommand{\fann}{{\forall n\in \nat}}
\newcommand{\pow}{{{\cal P}}}
\newcommand{\nll}{{{\bf NULL}}}
\newcommand{\nvc}[1]{{{\bf e_{#1}}}}
\newcommand{\st}{{\Sigma_{2}^{A}}}
\newcommand{\ov}[1]{{\overline{#1}}}
\newcommand{\provided}{{\hspace{.1in}:-\hspace{.1in}}}
\begin{document}
\begin{center}\title*{\Large \S \; 4.3 Primes \& GCD}\\\author*{Jessica Wei} \end{center}
\normalsize
\noindent
%-------------PRIME-------------------
\textbf{\subsection*{Prime}}
\begin{framed}
\noindent\textbf{DEF} $|$ Prime\\
Let $p\in\mathbb{Z}^+$. We say $p$ is a prime if it is only divisible by $p$ (itself) and 1. i.e. 2, 3, 5, 7...\\
Note: If an integer is not prime, then it is called composite.
\end{framed}
\begin{framed}
\noindent\textbf{THM 4.3.2} $|$ \\
Every composite integer has a prime factorization.
\[a = p_1^{a_1}\cdot p_2^{a_2}\cdot ... p_k^{a_k}\]
\end{framed}
%------------------EX 1
\noindent\textbf{Example 1.} Find the prime factorization of 100.\\
\[100 = 10 \cdot 10\]
\[10 = 2 \cdot 5\]
\[\therefore 100 = 2\cdot 2 \cdot 5 \cdot 5\]
\[\textbf{Answer: }100 = 2^2\cdot5^2\]

\end{document}