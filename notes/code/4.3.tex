\documentclass [12pt]{article}
\setlength{\oddsidemargin}{0.1in}
\setlength{\evensidemargin}{0.1in}
\setlength{\topmargin}{-.7in}
\setlength{\textheight}{9.25in}
\setlength{\textwidth}{6.5in}
\usepackage{caption}
\usepackage{subcaption}
\usepackage{enumerate}
\usepackage{framed}
\usepackage{epsfig}
\usepackage{changebar}
\usepackage{amsfonts}
\usepackage{amsmath}
\usepackage{amssymb}
\usepackage{graphicx}
\usepackage{amssymb}
\usepackage{stmaryrd}
\graphicspath{ {images/} }
\usepackage{listings}
\usepackage[usenames,dvipsnames]{color}
\usepackage{multicol}
\usepackage{mathtools}
\DeclarePairedDelimiter\ceil{\lceil}{\rceil}
\DeclarePairedDelimiter\floor{\lfloor}{\rfloor}
\setlength{\columnsep}{1cm}
% mathematical commands
\newcommand{\zos}{{\{ 0,1\}^{\ast}}}
\newcommand{\zoi}{{\{ 0,1\}^{\infty}}}
\newcommand{\zon}{{\{ 0,1\}}}
\newcommand{\zov}[1]{{\{ 0,1\}^{#1}}}
\newcommand{\ccc}{{{\cal C}}}
\newcommand{\gggg}{{{\cal G}}}
\newcommand{\nat}{{{\cal N}}}
\newcommand{\rr}{{{\bf RAND}}}
\newcommand{\pref}{{\sqsubset}}
\newcommand{\da}{{\downarrow}}
\newcommand{\ot}{{\otimes}}
\newcommand{\fann}{{\forall n\in \nat}}
\newcommand{\pow}{{{\cal P}}}
\newcommand{\nll}{{{\bf NULL}}}
\newcommand{\nvc}[1]{{{\bf e_{#1}}}}
\newcommand{\st}{{\Sigma_{2}^{A}}}
\newcommand{\ov}[1]{{\overline{#1}}}
\newcommand{\provided}{{\hspace{.1in}:-\hspace{.1in}}}
\begin{document}
\begin{center}\title*{\Large \S \; 4.3 Primes \& GCD}\\\author*{Jessica Wei} \end{center}
\normalsize
\noindent
%-------------PRIME-------------------
\textbf{\subsection*{Prime}}
\begin{framed}
\noindent\textbf{DEF} $|$ Prime\\
Let $p\in\mathbb{Z}^+$. We say $p$ is a prime if it is only divisible by $p$ (itself) and 1. i.e. 2, 3, 5, 7...\\
Note: If an integer is not prime, then it is called composite.
\end{framed}
%--------------------THM 4.3.1
\begin{framed}
\noindent\textbf{THM 4.3.1} $|$ \\
Every composite integer has a prime factorization.
\[a = p_1^{a_1}\cdot p_2^{a_2}\cdot ... p_k^{a_k}\]
\end{framed}
%------------------EX 1
\noindent\textbf{Example 1.} Find the prime factorization of 100.\\
\[100 = 10 \cdot 10\]
\[10 = 2 \cdot 5\]
\[\therefore 100 = 2\cdot 2 \cdot 5 \cdot 5\]
\[\textbf{Answer: }100 = 2^2\cdot5^2\]
%-----------------THM 4.3.2
\begin{framed}
\noindent\textbf{THM 4.3.2} $|$ \\
If $n$ is composite, then $\exists p\in\mathbb{Z}$such that $p$ is prime, $p\leq\sqrt{n}$, and $p|n$.\\\\
\textbf{PROOF:} Since n is composite, there exists $a,b\in\mathbb{Z}$ such that $n=a\cdot b$ and $a\neq n \neq b$.\\\\
Case 1: $a > \sqrt{n}$ and $b>\sqrt{n}$
\begin{enumerate}[\quad]
    \item Then $a\cdot b > \sqrt{n}$, $\sqrt{n} = n$
    \item $\therefore a \cdot b > n$
    \item This contradicts $a\cdot b = n$
    \item Hence, it must be true that either $a \leq \sqrt{n}$ or $b \leq \sqrt{n}$
\end{enumerate}
Assume $a\leq \sqrt{n}$. If this is not true, the following argument can be made for b.
\begin{enumerate}[\quad]
    \item Possibility \#1. $a$ is prime. Then $a\leq n$ satisfies and $n=a\cdot b\Rightarrow a|n$ satisfies.
    \item Possibility \#2. $a$ is composite. Then by THM 4.3.1, $a$ has a prime factorization so that $n=a\cdot b = p_1^{a_1}....p_x^{a_x}\cdot b$. SO any prime $p_i$ in this factorization divides n $(p_i|n)$ and $ p_i^{a_i}....p_x^{a_x} = a \leq \sqrt{n}$ which implies $p_i\leq \sqrt{n}$
\end{enumerate}
\end{framed}

%------------------EX 2
\noindent\textbf{Example 2.} Show that 61 is prime.\\
\vspace{0.1in}
\quad Assume that 61 is composite. In such a case $\exists p \leq \sqrt{61}$ such that $p|61$.
\begin{enumerate}[\quad]
    \item $\sqrt{61}\approx7...$, $p\leq\sqrt{61}\approx7...$
    \item Range of primes: 2, 3, 5, 7...
    \item $2\nmid61$, $3\nmid61$, $5\nmid61$, $7\nmid61$
    \item $\Rightarrow$ Contradiction hence 61 cannot be composite. 61 must be prime.
\end{enumerate}
%-----------------THM 4.3.3
\begin{framed}
\noindent\textbf{THM 4.3.3} $|$ \\
If $n$ is composite, then $\exists p\in\mathbb{Z}$such that $p$ is prime, $p\leq\sqrt{n}$, and $p|n$.\\\\
\textbf{PROOF:} Since n is composite, there exists $a,b\in\mathbb{Z}$ such that $n=a\cdot b$ and $a\neq n \neq b$.\\\\
Case 1: $a > \sqrt{n}$ and $b>\sqrt{n}$
\begin{enumerate}[\quad]
    \item Then $a\cdot b > \sqrt{n}$, $\sqrt{n} = n$
    \item $\therefore a \cdot b > n$
    \item This contradicts $a\cdot b = n$
    \item Hence, it must be true that either $a \leq \sqrt{n}$ or $b \leq \sqrt{n}$
\end{enumerate}
Assume $a\leq \sqrt{n}$. If this is not true, the following argument can be made for b.
\begin{enumerate}[\quad]
    \item Possibility \#1. $a$ is prime. Then $a\leq n$ satisfies and $n=a\cdot b\Rightarrow a|n$ satisfies.
    \item Possibility \#2. $a$ is composite. Then by THM 4.3.1, $a$ has a prime factorization so that $n=a\cdot b = p_1^{a_1}....p_x^{a_x}\cdot b$. SO any prime $p_i$ in this factorization divides n $(p_i|n)$ and $ p_i^{a_i}....p_x^{a_x} = a \leq \sqrt{n}$ which implies $p_i\leq \sqrt{n}$
\end{enumerate}
\end{framed}
\end{document}