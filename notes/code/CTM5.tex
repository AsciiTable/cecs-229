\documentclass [12pt]{article}
\setlength{\oddsidemargin}{0.1in}
\setlength{\evensidemargin}{0.1in}
\setlength{\topmargin}{-.7in}
\setlength{\textheight}{9.25in}
\setlength{\textwidth}{6.5in}
\usepackage{caption}
\usepackage{subcaption}
\usepackage{enumerate}
\usepackage{framed}
\usepackage{epsfig}
\usepackage{changebar}
\usepackage{amsfonts}
\usepackage{amsmath}
\usepackage{amssymb}
\usepackage{graphicx}
\usepackage{amssymb}
\usepackage{stmaryrd}
\graphicspath{ {images/} }
\usepackage{listings}
\usepackage[usenames,dvipsnames]{color}
\usepackage{multicol}
\usepackage{mathtools}
\DeclarePairedDelimiter\ceil{\lceil}{\rceil}
\DeclarePairedDelimiter\floor{\lfloor}{\rfloor}
\setlength{\columnsep}{1cm}
% mathematical commands
\newcommand{\zos}{{\{ 0,1\}^{\ast}}}
\newcommand{\zoi}{{\{ 0,1\}^{\infty}}}
\newcommand{\zon}{{\{ 0,1\}}}
\newcommand{\zov}[1]{{\{ 0,1\}^{#1}}}
\newcommand{\ccc}{{{\cal C}}}
\newcommand{\gggg}{{{\cal G}}}
\newcommand{\nat}{{{\cal N}}}
\newcommand{\rr}{{{\bf RAND}}}
\newcommand{\pref}{{\sqsubset}}
\newcommand{\da}{{\downarrow}}
\newcommand{\ot}{{\otimes}}
\newcommand{\fann}{{\forall n\in \nat}}
\newcommand{\pow}{{{\cal P}}}
\newcommand{\nll}{{{\bf NULL}}}
\newcommand{\nvc}[1]{{{\bf e_{#1}}}}
\newcommand{\st}{{\Sigma_{2}^{A}}}
\newcommand{\ov}[1]{{\overline{#1}}}
\newcommand{\provided}{{\hspace{.1in}:-\hspace{.1in}}}
\begin{document}
\begin{center}\title*{\Large \S \; 5 Gaussian Elimination I \& II}\\\author*{Jessica Wei} \end{center}
\normalsize
\noindent\textbf{MOTIVATION}: Grow Algorithm needs a solution to a system 
\[A\overrightarrow{v}=\overrightarrow{b}\indent A\in\mathbb{F}_{m\times n}\indent \overrightarrow{b}\in\mathbb{F}^m\]
$\overrightarrow{x}=A^{-1}\overrightarrow{b}$ works only if $m=n$ (square matrix)\\
\begin{framed}
\noindent\textbf{DEF} $|$ Augmented Matrix\\
Given the problem $A\overrightarrow{x}=\overrightarrow{b},A\in\mathbb{F}_{m\times n}$ the augmented matrix for this system\\
$A_G=\begin{bmatrix}
A&.&1
\end{bmatrix}=\begin{bmatrix}
1
\end{bmatrix}$
\end{framed}
\noindent\textbf{Example. }Form $A_G$ for the system
\[x+2y+3z=7\]
\[4x+5y+8z=8\]
$\begin{bmatrix}
1&2&3\\
4&5&6
\end{bmatrix}\begin{bmatrix}
x\\
y\\
z
\end{bmatrix}=\begin{bmatrix}
7\\
8
\end{bmatrix}$\\\\
$A_G=\begin{bmatrix}
1&2&3&7\\
4&5&8&8
\end{bmatrix}$
\begin{framed}
\noindent\textbf{DEF} $|$ Reduced Row-Echelon Form (RREF)\\
Let $A_G,B\in\mathbb{F}_{m\times (n+1)}$ where $A_G$ is the augmented matrix of the system $A\overrightarrow{x}=\overrightarrow{b}, A\in\mathbb{F}_{m\times n}$
\end{framed}
We call matrix $B$ the RREF($A_G$) if 
\begin{enumerate}[(1)]
    \item $\begin{bmatrix}
    b_{11}&...&b_{1n}\\
    .\\
    .\\
    b_{m1}&...&b_{mn}
    \end{bmatrix}\begin{bmatrix}
    x_1\\
    .\\
    .\\
    x_n
    \end{bmatrix}=\begin{bmatrix}
    b_{1n+1}\\
    .\\
    .\\
    b_{mn+1}
    \end{bmatrix}$ and
    \item if $n(i)=$column index of first nonzero entry of row $i$\\
    then $n(i)<n(k)$ whenever $i<k$.
\end{enumerate}
\noindent\textbf{Example. }Find the RREF for the augmented matrix of the system
\[4x+y=5\]
\[2x-y=1\]
$\begin{bmatrix}
4&1\\
2&-1
\end{bmatrix}$ $\begin{bmatrix}
    x\\
    y
\end{bmatrix}$
= $\begin{bmatrix}
    5\\
    1
\end{bmatrix}$\\\\
$A_G=\begin{bmatrix}
4&1&5\\
2&-1&1
\end{bmatrix}$\\\\
GOAL: $RREF(A_G)=\begin{bmatrix}
*&*&*\\
0&*&*
\end{bmatrix}$\\\\
Notice if vector $[x,y]^T$ that solves the system
\[[4,1]\cdot[x,y]=5\]
\[[2,-1]\cdot[x,y]=1\]
Then,
\[[4,1]\cdot[x,y] - 2[2,-1]\cdot[x,y] = 5-2\]
In such case, 
\[4x+y-2(2x-y)=3\]
\[4x+y-4x+2y=3\]
\[3y=3\]
\[[0,3]\cdot[x,y]=3\]
i.e. Solving the system 
\[[4,1]\cdot[x,y]=5\]
\[[2,-1]\cdot[x,y]=1\]
is equivalent to solving the system 
\[[4,1]\cdot[x,y]=5\Rightarrow 4x+y=5\]
\[[0,3]\cdot[x,y]=3\Rightarrow 3y=3\]
$\Rightarrow B=\begin{bmatrix}
4&1&5\\
0&3&3
\end{bmatrix}\Rightarrow$ RREF($A_G$) = $B$\\\\\\\\
\noindent\textbf{*Remarks}
\begin{enumerate}[\quad(i)]
    \item In general, we can construct RREF($A_G$) by adding multiples of row vectors to form rows with zeros in the appropriate columns.
    \item If a row contains all zero entries in a RREF matrix, then all rows below it must be zeros as well: $n(i)<n(k)$ for $i<k$
    \item The nonzero rows of a RREF matrix form a basis for the for the row space of $A_G$.
    \item If $A_G=\begin{bmatrix}
    a_{11}&...&a_{1n}&b_1\\
    .\\
    .\\
    a_{m1}&...&a_{mn}&b_n
    \end{bmatrix}$ has RREF($A_G$)$=\begin{bmatrix}
    r_{11}&...&r_{1n}&\hat{b}_1\\
    0&r_{22}&...&r_{2n}&\hat{b}_2
    .\\
    .\\
    0&...&...&r_{nn}&\hat{b})n
    \end{bmatrix}$\\\\
    Then we can  obtain the solution vector $\overrightarrow{x}$ to the system
    \[a_{11}x_1+...+a_{1n}x_n=b_1\]
    \[...\]
    \[a_{m1}x_1+...+a_{mn}x_n=b_n\]
\end{enumerate}
\noindent\textbf{Example. }Find the RREF of the system \& use it to find a solution
\[x+3y=4\]
\[2x-y=1\]
\[3x+2y=5\]
\[5x+15y=20\]
\\\\
$A_G=\begin{bmatrix}
1&3&4\\
2&-1&1\\
3&2&5\\
5&15&20
\end{bmatrix} \Rightarrow \begin{bmatrix}
1&3\\
2&-1\\
3&2\\
5&15
\end{bmatrix}\begin{bmatrix}
x\\
y
\end{bmatrix}=\begin{bmatrix}
4\\
1\\
5\\
20
\end{bmatrix}$\\\\
$A_G=\begin{bmatrix}
1&3&4\\
2&-1&1\\
3&2&5\\
5&15&20
\end{bmatrix}\Rightarrow\begin{bmatrix}
r_1\\
r_2=r_2-2r_1\\
r_3=r_3-3r_1\\
r_4=r_4-5r_1
\end{bmatrix} \Rightarrow \begin{bmatrix}
1&3&4\\
0&-7&-7\\
0&-7&-7\\
0&0&0
\end{bmatrix}\indent$\\ $r_3=r_3-r_2\Rightarrow\begin{bmatrix}
1&3&4\\
0&-7&-7\\
0&0&0\\
0&0&0
\end{bmatrix}\Rightarrow\begin{matrix}
\rightarrow&x+3y=4\\
\rightarrow-7y=-7\Rightarrow y=1, x=4-3y=1\\
\end{matrix}$ \\\\
$\Rightarrow\begin{bmatrix}
x\\
y
\end{bmatrix}=\begin{bmatrix}
1\\
1
\end{bmatrix}$\\\\
\begin{framed}
\noindent\textbf{Procedure. }Gaussian Elimination\\
IN: Matrix $A\in\mathbb{F}_{m\times n}$, vector $\overrightarrow{b}\in\mathbb{F}^n$\\
OUT: Vector $\overrightarrow{x}\in\mathbb{F}^m$ solution to $A\overrightarrow{x}=\overrightarrow{b}$\\
Preliminary Step: Form $A_G$.\\
\textbf{Step 1:} For every row, $r_i$, $i=2,3,...,m$, replace $\overrightarrow{r_i}=\overrightarrow{r_i}-(\frac{r_{i1}}{11})\overrightarrow{r_1}$\\
For every row, $\overrightarrow{r}_i$, $i=2,3,...,m$, replace $\overrightarrow{r}_i=\overrightarrow{r}_i-(\frac{r_{i1}}{a_{11}})\overrightarrow{r}_1$\\
$\begin{bmatrix}
a_{11}&a_{12}&...&a_{1n}&b_1\\
0&a_{22}&...&a_{2n}^{(1)}&b_2^{(1)}\\
0&a_{32}^{(1)}&...&a_{3n}^{(1)}&b_3^{(1)}\\
.\\
.\\
0&a_{m2}&...&a_{mn}^{(1)}&b_m^{(1)}
\end{bmatrix}\indent\begin{matrix} \overrightarrow{r}_2=\overrightarrow{r}_2-(\frac{a_{21}}{a_11})\overrightarrow{r}_1\\
\overrightarrow{r}_3=\overrightarrow{r}_3-(\frac{a_{21}}{a_11})\overrightarrow{r}_1\\
\overrightarrow{r}_m=\overrightarrow{r}_m-(\frac{a_{m1}}{a_11})\overrightarrow{r}_1
\end{matrix}$\\\\
\textbf{Step 2:} Reducing at 3rd row. For every row, $\overrightarrow{r}_i$, $i=3,4,...,m$, replace $\overrightarrow{r}_i=\overrightarrow{r}_i-(\frac{a_{i2}^{(1)}}{a_{22}^{(1)}})\overrightarrow{r}_2$\\
\textbf{Step k:} Reducing at row $(k+1)$. For every row, $\overrightarrow{r}_i$, $i=k+1,k+2...m$, replace $\overrightarrow{r}_i=\overrightarrow{r}_i-(\frac{a_{ik}^{(k-1)}}{a_{kk}^{(k-1)}})$\\
Repeat until step $k=m-1$.\\
\textbf{Final Step:} Use the RREF($A_G$) to solve for $\overrightarrow{x}$ using \textbf{Backward Substitution}.\\
$$\begin{bmatrix}
a_{11}&a_{12}&...&...&a_{1n}&b_1\\
0&a_{22}&...&...&a_{2n}^{(1)}&b_2^{(1)}\\
0&0&a_{33}^{(1)}&...&a_{3n}^{(1)}&b_3^{(1)}\\
.\\
.\\
0&0&...&...&a_{mn}^{(k)}&b_m^{(k)}
\end{bmatrix}$$\\
If the last nonzero row is row $n$, then (i.e. $rank(A_G)=rank(A)=n=$ \# of columns in $A$)
\[*a_{nn}x_n=b_n\Rightarrow x_n=\frac{b_n}{a_{mn}}\]
\[*a_{n-1n-1}x_{n-1}+a_{n-1n}x_{n}=b_{n-1}\Rightarrow x_{n-1} = \frac{1}{a_{n-1n-1}}(b_{n-1}-a_{n-1n}x_{n})\]
\[*a_{kk}x_k+a_{kk-1}x_{k-1}+...+a_{kn}x_{n}=b_k\]
\[\Rightarrow x_k=\frac{1}{a_{kk}}[b_k-a_{kk-1}x_{k-1}-...-a_{kn}x_n]\]
\end{framed}
\noindent\textbf{Example. } Use Gaussian Elimination to solve \\
\[2c+3d=4 \Rightarrow  a+b+2c=0\]
\[a+3c+d=2 \Rightarrow a+3c+d=2\]
\[a+b+2c=0 \Rightarrow 2c+3d=4\]
$$\begin{bmatrix}
1&1&2&0&0\\
1&0&3&1&2\\
0&0&2&3
\end{bmatrix} = \begin{bmatrix}
1&1&2&0&0\\
0&-1&1&1&2\\
0&0&2&3&4
\end{bmatrix}$$
\[a+b+2c=0\]
\[-b+c+d=2\]
\[2c+3d=4\]
Let $d$ be a free variable.\\
From $2c+3d=4$: $c=\frac{1}{2}[4-3d]\indent c=2-\frac{3}{2}d$\\
From $-b+c+d=2$: $b=c+d-2=\frac{3}{2}d+d\indent b=\frac{-1}{2}d$\\
From $a+b+2c=0$: $a=-b-2x=-(\frac{-1}{2}d)-2(2-\frac{3}{2}d)=\frac{1}{2}d-3d\indent a=\frac{7}{2}d-4$
\[=[\frac{7}{2}d-4,\frac{-1}{2}d,2-4d,d]\]
$$\begin{bmatrix}
1&1&2&0\\
1&0&3&1\\
0&0&2&3
\end{bmatrix}\begin{bmatrix}
a\\
b\\
c\\
d
\end{bmatrix}=\begin{bmatrix}
4\\
2\\
0
\end{bmatrix}$$\\
Therefore there is NO UNIQUE SOLUTION (infinitely many)\\\\
\noindent\textbf{Example. } Use Use Gaussian Elimination to solve
\[2x+4y+3z=8\]
\[3x-4y-4z=3\]
\[5x-z=12\]
$A_G=\begin{bmatrix}
2&4&3&8\\
3&-4&-4&3\\
5&0&-1&12
\end{bmatrix}\Rightarrow\begin{bmatrix}
2&4&3&8\\
0&-10&\frac{-17}{2}&-9\\
0&-10&\frac{-17}{2}&-8
\end{bmatrix}\Rightarrow\begin{bmatrix}
2&4&3&8\\
0&-10&\frac{-17}{2}&-9\\
0&0&0&1
\end{bmatrix}$\\\\
At this point, we see an issue: last two rows form the system 
\[-10x-\frac{-17}{2}z=-9\]
\[-10y-\frac{-17}{2}z=-8\]
No two numbers satisfy both equations at the same time.\\
row 3: $0x+0y+0z=1\indent 0\neq 1$\\ 
NOT POSSIBLE $\Rightarrow$ No Solution

\begin{framed}
\noindent\textbf{Summary. }Given the problem: $A\overrightarrow{x}=\overrightarrow{b}$ where $A\in\mathbb{F}_{m\times n},\overrightarrow{b}\in\mathbb{F}^m$...
\begin{enumerate}[\indent1.]
    \item Form augmented matrix $A_G\in\mathbb{F}_{m\times (n+1)}$
    \item Find RREF($A_G$):\\
    \indent CASE 1: $rank(A_G)=n\Rightarrow$ Perform Backward Substitution to obtain a unique solution \\
    \indent CASE 2: $rank(A_G)<n \& r_{ki}\neq 0$ for some where $k$ is the last nonzero row. In such a case $\Rightarrow$ infinitely many solutions where $r_{k,k+1},r_{k,k+2}...r_{kn}$ being free variables (or some subset of \{$r_{k,k},r_{k,k+1},...,r_{kn}$\})\\
    \indent CASE 3: $rank(A_G)<n$ and $r_{ki}=0$ for $i=k,k+1,...,n\Rightarrow0=b_k$ has NO SOLUTION
\end{enumerate}
\end{framed}
\pagebreak

\noindent\textbf{GAUSSIAN ELIMINATION WITH PIVOTING}\\\\
\noindent\textbf{Motivation:} Suppose you program a function to solve $A\overrightarrow{x}=\overrightarrow{b}$ and you test it with \\
$A=\begin{bmatrix}
10^{-20}&0\\
1&10^{20}\\
0&1
\end{bmatrix}\indent\overrightarrow{b}=\begin{bmatrix}
1\\
1\\
-1
\end{bmatrix}$\\\\
Expectation:\\
$A_G=\begin{bmatrix}
10^{20}&0&1\\
1&10^{20}&1\\
0&1&-1
\end{bmatrix}\Rightarrow\begin{bmatrix}
10^{-20}&0&1\\
0&10^{20}&1-10^{20}\\
0&1&-1
\end{bmatrix}\Rightarrow\begin{bmatrix}
10^{-20}&0&1\\
0&10^{20}&1-10^{20}\\
0&0&-10^{-20}
\end{bmatrix}$
\[0x+0y=-10^{-20}\]
$\Rightarrow $ NO SOLUTION \\
What Actually Happens:\\
$A_G=\begin{bmatrix}
10^{20}&0&1\\
1&10^{20}&1\\
0&1&-1
\end{bmatrix}=\begin{bmatrix}
10^{-20}&0&1\\
0&10^{20}&-10^{20}\\
0&1&-1
\end{bmatrix}=\begin{bmatrix}
10^{-20}&0&1\\
0&10^{20}&-10^{20}\\
0&0&0
\end{bmatrix}$
\[\Rightarrow10^{20}y=-10^{20}\Rightarrow y=-1\]
\[\Rightarrow 10^{-20}x=1\Rightarrow x=10^{20}\]
$\Rightarrow$ Unique Solution: $[10^{20}, -1]$\\
What went wrong? When we formed the augmented matrix $A_G=\begin{bmatrix}
10^{20}&0&1\\
1&10^{20}&1\\
0&1&-1
\end{bmatrix}$, our pivot $10^{-20}$ was small in comparison with the rest of the values (the same is true if the pivot is extremely large). In this case, reducing row $i$ by 
\[r_i=r_i-\frac{r_{i1}}{r_{11}}\overrightarrow{r}_1\] cases $\frac{r_{i1}}{r_11}$ to be extremely large (or small) causing round-off errors in the process. Where $r_{11}=$ entry in row1, col1 and $\overrightarrow{r}_1=$ row1.\\

\begin{framed}
\noindent\textbf{DEF} $|$ Pivot\\
The pivot is the first nonzero element of a row that will be used to reduce all other rows below it.
$\begin{bmatrix}
a_{11}&a_{12}&...&a_{1n}\\
0&a_{22}&...&a_{2n}\\
0&a_{32}&...&a_{3n}\\
.\\
.\\
0&a_{m2}&...&a_{mn}
\end{bmatrix}$
$a_{22}$ is the pivot\\
*If the pivot is very small, then severe round-off errors are introduced into the calculations. This potentially yields an incorrect answer. 
\end{framed}
\pagebreak
\noindent\textbf{METHOD 1.} Gaussian Elimination with Partial Pivoting.
\begin{enumerate}[\quad]
    \item Step 0: Form the augmented matrix of the system.
    \item Step K: *Swap rows so that the pivot is the element with the largest absolute value in its column. *proceed to reduce the rows using this pivot.
    \item Step n: By the beginning of this step, you should have a matrix RREF($A_G$). Use Backward Substitution to solve the system. 
\end{enumerate}
\noindent\textbf{Example.} Solve
\[x+2y+4z=18\]
\[2x+12y-2z=9\]
\[5x+26y+5z=14\]
Using Gaussian Elimination with Partial Pivoting.\\
Step 0: $A_G = \begin{bmatrix}
1&2&4&18\\
2&12&-2&9\\
5&26&5&14
\end{bmatrix}$\\\\
Step 1: Swap Row 0 and 2\\
$A_G = \begin{bmatrix}
5&26&5&14\\
2&12&-2&9\\
1&2&4&18
\end{bmatrix}\Rightarrow \begin{bmatrix}
5&26&5&14\\
0&1.6&-4&3.4\\
0&-3.2&3&15.2
\end{bmatrix}$\\\\
Step 2: Swap Row 1 and 2 b/c -3.2's absolute value is greater than 1.6\\
$A_G=\begin{bmatrix}
5&26&5&14\\
0&-3.2&3&15.2\\
0&1.6&-4&3.4
\end{bmatrix}\Rightarrow\begin{bmatrix}
5&26&5&14\\
0&-3.2&3&15.2\\
0&0&-2.5&11
\end{bmatrix}$\\\\
Step 3: Backward Substitution
\[-2.5z = 1\Rightarrow z= 4.4\]
\[-3.2y+3z=15.2\Rightarrow y = -8.875\]
\[5x=14-26y-5z\Rightarrow x=53.35\]
Answer: $[53.35,-8.875,-4.4]$\\\\\\
\noindent\textbf{METHOD 2.} Gaussian Elimination with Total Pivoting
\begin{enumerate}[\quad]
    \item Step 0: *Form the augmented matrix $A_G$\\
        *Add a row at the top that enumerates the columns, beginning at 1, 2, 3, ..., $n-1$, 0 where 0 is the index of the column representing the constants.
    \item Step K: *Find the element with the largest absolute value in the submatrix consisting of rows that will be reduced (do not consider constants i.e. elements in column 0). Swap rows and columns so that such element becomes the pivot.\\
    *Reduce the rows using this pivot until you reach step $n-1$.
    \item Step n: *At this point, you should have RREF($A_G$), which you can solve using Backward Substitution. 
\end{enumerate}
\pagebreak
\noindent\textbf{Example. }Solve
\[x+2y+4z=18\]
\[2x+12y-2z=9\]
\[5x+26y+5z=14\]
Using Gaussian Elimination with Total Pivoting.\\
Step 0: Swap row 1 with 3 and column 1 with 0 (26 is our pivot.)\\
$A_G = \begin{bmatrix}
1&2&3&0\\
1&2&4&18\\
2&12&-2&9\\
5&26&5&14
\end{bmatrix}\Rightarrow \begin{bmatrix}
1&2&3&0\\
5&26&5&14\\
2&12&-2&9\\
1&2&4&18
\end{bmatrix}\Rightarrow\begin{bmatrix}
2&1&3&0\\
26&5&5&14\\
12&2&-2&9\\
2&1&4&18
\end{bmatrix}\\\Rightarrow \begin{bmatrix}
2&1&3&0\\
26&5&5&14\\
0&-0.3077&-4.3077&2.5385\\
0&0.6154&3.6154&16.9231
\end{bmatrix}$\\\\
Step 2: Swap columns 1 and 3\\
$A_G = \begin{bmatrix}
2&3&1&0\\
26&5&5&14\\
0&-4.3077&-0.3077&2.5385\\
0&3.6154&0.6154&16.9231
\end{bmatrix}\Rightarrow\begin{bmatrix}
2&3&1&0\\
26&5&5&14\\
0&-4.3077&-0.3077&2.5385\\
0&0&0.3571&19.0536
\end{bmatrix}$\\\\
Step 3: Backward Substitution
\[r_3: 0.3571x=10.0536\Rightarrow x=53.35\]
\[r_2: -4.3077z-0.3077x=2.5385\Rightarrow z=-4.3985\]
\[r_1: 26y+5z+5x=14\Rightarrow y = -8.8749\]
Answer. $[53.35, -8.8749, -4.3985]$\\\\
\textbf{Gaussian Elimination with Partial or Total Pivoting}\\
Yields a solution to $A\overrightarrow{x}=\overrightarrow{b}$ with MINIMIZED ROUND-OFF ERROR... But there might still be some error.\\
\textbf{Goal.} Define a way to measure that error. \\
Let $\overrightarrow{x}$ be the vector obtained from applying G.E.P.P or G.E.T.P.
\[A\overrightarrow{x}\approx \overrightarrow{b}\]
\[\overrightarrow{b}^k\approx \overrightarrow{b}\]
\[error=|\overrightarrow{b}^k-\overrightarrow{b}|\]
This is not defined for vectors yet.\\
\begin{framed}
\textbf{DEF} $|$ Norm\\
Let $\overrightarrow{x}\in\mathbb{R}^n$\\
\[*L_1-norm: ||\overrightarrow{x}||_1=|x_1|+x_2|+...+|x_n|\]
e.g. $\overrightarrow{x}=[-1.2,-3,4]\quad ||\overrightarrow{x}||=|-1.2|+3+4=8.2$
\[*L_2-norm: ||\overrightarrow{x}||_2=\sqrt{x_1^2+...+x_n^2}\]
\[L_p-normp:||\overrightarrow{x}||_p=[|x_1|^p+...+|x_n|^p]^{\frac{1}{p}}\]
So we can find error by taking a norm of the difference between vector $\overrightarrow{b^*}$ and $\overrightarrow{b}$.\\
E.g. $A\overrightarrow{x}=\begin{bmatrix}
1.01\\
2.25\\
-1.13
\end{bmatrix}=\overrightarrow{b}^*$\\
$\overrightarrow{b}=\begin{bmatrix}
1.02\\
2.24\\
-1.1
\end{bmatrix}$\\\\
$\overrightarrow{diff}=\overrightarrow{b}-\overrightarrow{b}^*=[0.01,-0.01,-0.03]\Rightarrow$ error under $L_1$ norm: $||\overrightarrow{diff}||_1$
\[=|0.01|+|-0.01|+|-0.03|=0.05\]
error under $L_2$ norm: $||\overrightarrow{diff}||_2$
\[=\sqrt{(0.01)^2+(-0.01)^2+(-0.03)^2}=0.03\]
*You choose the norm depending on your situation. The given error then can provide a means for accepting a solution. e.g. if error is too high, you may deem $\overrightarrow{x}$ solution as not good enough. 
\end{framed}
\end{document}