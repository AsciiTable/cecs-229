\documentclass [12pt]{article}
\setlength{\oddsidemargin}{0.1in}
\setlength{\evensidemargin}{0.1in}
\setlength{\topmargin}{-.7in}
\setlength{\textheight}{9.25in}
\setlength{\textwidth}{6.5in}
\usepackage{caption}
\usepackage{subcaption}
\usepackage{enumerate}
\usepackage{framed}
\usepackage{epsfig}
\usepackage{changebar}
\usepackage{amsfonts}
\usepackage{amsmath}
\usepackage{amssymb}
\usepackage{graphicx}
\usepackage{amssymb}
\usepackage{stmaryrd}
\graphicspath{ {images/} }
\usepackage{listings}
\usepackage[usenames,dvipsnames]{color}
\usepackage{multicol}
\usepackage{mathtools}
\DeclarePairedDelimiter\ceil{\lceil}{\rceil}
\DeclarePairedDelimiter\floor{\lfloor}{\rfloor}
\setlength{\columnsep}{1cm}
% mathematical commands
\newcommand{\zos}{{\{ 0,1\}^{\ast}}}
\newcommand{\zoi}{{\{ 0,1\}^{\infty}}}
\newcommand{\zon}{{\{ 0,1\}}}
\newcommand{\zov}[1]{{\{ 0,1\}^{#1}}}
\newcommand{\ccc}{{{\cal C}}}
\newcommand{\gggg}{{{\cal G}}}
\newcommand{\nat}{{{\cal N}}}
\newcommand{\rr}{{{\bf RAND}}}
\newcommand{\pref}{{\sqsubset}}
\newcommand{\da}{{\downarrow}}
\newcommand{\ot}{{\otimes}}
\newcommand{\fann}{{\forall n\in \nat}}
\newcommand{\pow}{{{\cal P}}}
\newcommand{\nll}{{{\bf NULL}}}
\newcommand{\nvc}[1]{{{\bf e_{#1}}}}
\newcommand{\st}{{\Sigma_{2}^{A}}}
\newcommand{\ov}[1]{{\overline{#1}}}
\newcommand{\provided}{{\hspace{.1in}:-\hspace{.1in}}}
\begin{document}
\begin{center}\title*{\Large \S \; 1.1 - 1.5 Intro to Fields}\\\author*{Jessica Wei} \end{center}
\normalsize
\noindent
%-------------Fields
\textbf{\subsection*{Fields}}
\begin{framed}
\textbf{DEF} $|$ Field\\
A field F is a set of values with defined operations for addition and multiplication. The operations satisfy the following:
\begin{enumerate}[\quad(i)]
    \item Closure: for $x, y\in\mathbb{Z}$\\
    \quad $x+y\in F$\\
    \quad $xy\in F$
    \item Commutative\\
    \quad $x+y = y + x$\\
    \quad $xy = yx$
    \item Associative; $z\in\mathbb{F}$\\
    \quad $x+y + z = x + (y + z)$\\
    \quad $(xy)z = x(yz)$
    \item Identity element I\\
    \quad $x+ I_+ = I_+ + x = x$ for any element x\\
    \quad $x\cdot I_+ = I_x\cdot x = x$
    \item Inverse Element\\
    \quad $x+\Bar{x} = I_+$ for every $x\in F$ \\
    \quad $x\cdot \hat{x} = I_x$
    \item Distributive Identity: $x,y,z\in F$\\
    \quad $x(y+z) = xy + xz$
\end{enumerate}
\end{framed}
\vspace{0.2in}

\noindent\textbf{Example 1.} Verify that the set of real numbers $\mathbb{R}$ is a field.\\
\begin{enumerate}[\quad(i)]
    \item For any two real numbers $x,y\in\mathbb{R}$\\
    \quad $x+y\in\mathbb{R}$, $xy\in\mathbb{R}$
    \item $x+y = y+x$ is TRUE\\
    \quad$xy = yx$ is TRUE
    \item Associativity\\
    \quad $(x+y) + z = x+(y + z)$
    \item Identity\\
    \quad $x+0=0+x=x$ \quad 0 is identity element under $I_+$\\
    \quad $x\cdot 1=1\cdot x=x$\quad $I_x = 1$
    \item Inverse\\
    \quad $x+(-x)=0\quad$ where $-x\in\mathbb{R}$ is the additive inverse\\
    \quad $x(\frac{1}{x}) = 1$ \quad $\frac{1}{x}\in\mathbb{R}$ is the multiplicative inverse as long as $x\neq 0$
    \item Distributive\\
    \quad $x(y+z) = xy+xz$ \quad i.e. $2(3+4) = 2(7) = 2\cdot3 + 2\cdot4 = 14$
\end{enumerate}
\vspace{0.2in}
\noindent\textbf{Example 2.} Is $\mathbb{Z}$ a field under the usual definitions of $+$ and $\cdot$?\\
\indent No. For any integer $a\in\mathbb{Z}$, it is not possible to find a multiplicative inverse $a \in\mathbb{Z}$ such that $a\hat{a} = 1$.
\vspace{0.2in}
\begin{framed}
\textbf{DEF} $|$ Complex Numbers\\
The set of complex numbers $C$ is the set of numbers of the form 
\[a+bi\]
where $a$ is the real part, $bi$ is the imaginary part, and $i^2=-1$.\\
*Note $\mathbb{C}$ is a field under the following operations:\\
\begin{enumerate}[\quad]
    \item For $z_1,z_2\in\mathbb{C}$:\\
    \quad Addition: $z_1+z_2 = (a_1+b_i) + (a_2+b_2i) = (a_1+a_2) + (b_1+b_2)i$\\
    \quad Multiplication: $z_1z_2=(a_1+b_1i)(a_2+b_2i)=a_1a_2+a_1b_2i+a_2b_1i+b_1b_2i^2$\\
    \quad $= (a_1a_2 - b_1b_2) + (a_1b_2+a_2b_1)i$
\end{enumerate}
\end{framed}
\vspace{0.2in}
\noindent\textbf{Example 3.} Perform the operation 
\begin{enumerate}[\quad(a)]
    \item $(3+2i)+(1+i)=(3+1)+(2+1)i=4+3i$
    \item $(3+2i)(7-5i)=21-15i+14i+10=31- i$
    \item $(3+2i)(3-2i)=9-6i+6i+4=13$
    \item $\frac{31-i}{7-5i} \cdot \frac{7-5i}{7-5i} = \frac{(217+155i-7i-5)}{49 + 25}=\frac{222+148i}{74} = \frac{222}{74} + \frac{148}{74}i = 3+2i$
\end{enumerate}
\pagebreak
\textbf{\subsection*{Visualizing Complex Numbers}}
\noindent Imaginary is on the y axis, real is on the x axis. We represent $3+2i$ as a vector $\overrightarrow{z}$ on the coordinate plane. Suppose that $z=a+bi$ is any complex number. \\
\indent $\Rightarrow tan\theta=\frac{b}{a}$\\
\indent $\Rightarrow |\overrightarrow{z}|=\sqrt{a^2+b^2} = \sqrt{z\Bar{z}}$\\
\indent Notice that $(a+bi)(a-bi) = a^2+b^2$ and $\Bar{z}$ is conjugate\\
$cos\theta=\frac{a}{|\overrightarrow{z}|}$, $sin\theta=\frac{b}{|\overrightarrow{z}|}$\\
$cos^2\theta=\frac{a^2}{|\overrightarrow{z}|^2}$, $sin^2\theta=\frac{b^2}{|\overrightarrow{z}|^2}$\\
\begin{enumerate}
    \item $1=cos^2\theta + sin^2\theta = \frac{a^2}{|\overrightarrow{z}|^2} + \frac{b^2}{|\overrightarrow{z}|^2}$\\
    $\Rightarrow a^2+b^2 = |\overrightarrow{z}|^2$
    \item $a=|\overrightarrow{z}|cos\theta$, $b=|\overrightarrow{z}|sin\theta$\\
    $z=a+bi = |\overrightarrow{z}|cos\theta + |\overrightarrow{z}|sin\theta i$\\
    $=|\overrightarrow{z}|(cos\theta + sin\theta)$
\end{enumerate}
Euler's Identity: $e^{ie}=cos\theta+sin\theta$\\
\[\Rightarrow z=|\overrightarrow{z}|e^{i\theta}\]
where $|\overrightarrow{z}|=\sqrt{a^2+b^2}$ and $tan\theta=\frac{b}{a}$\\
$\Rightarrow z=re^{i\theta}$

\textbf{\subsection*{Translations}}
\begin{framed}
    \noindent\textbf{DEF} $|$ Translations\\
    Define $f(z) = z+ z_1$ where $z_1=a_1+b_1i$ is given (constant). Then $f(z)$ describes a translation: 
    \[f(z)=z+z_1\]
    \[=(a+bi)+(a_1b_1i)\]
    \[=(a+a_1)+(b+b_1)i\]
e.g. $a_1 = 2$, $b_1 = 3$\\
Add $a_1,b_1$ to original values of $a,b$ on the graph. In this case, the point/vector representing $z=a+bi$ has shifted \textbf{2 units right} and \textbf{3 units up}.\\
\vspace{0.1in}

\noindent In general for $z_1 = a_1+b_1i$, we have the following cases:\\
$a_1>0\quad$ shift right\\
$a_1 = 0\quad$ no horizontal shift\\
$a_1 < 0\quad$ shift left\\

\noindent$b_1>0\quad$ shift up\\
$b_1 = 0\quad$ no vertical shift\\
$b_1 < 0\quad$ shift down\\
\end{framed}
Let $f_1(z)=z+z_1\&f_2(z)=z+z_2,z_1,z_2\in\mathbb{C}$ given"
\[f_1of_2=f_1(f_2(z))=f_1(z+z_2)=z+z_2+z_1\]
\[=z+w\]
where $w = z_2+z_1$. Also describes a translation and $f_2of_1=f_1of_2$

\textbf{\subsection*{Scaling \& Rotation}}
\begin{framed}
\noindent\textbf{DEF} $|$ Scaling 
Define $h(z)=\alpha z$ where $\alpha\in\mathbb{R}$\\
Notice $|h(z)|=|\alpha z|=|\alpha(a+bi)|  = |a\alpha+b\alpha i|$\\
\[=\sqrt{(a\alpha)^2+(b\alpha)^2}\]
\[=\sqrt{a^2\alpha^2+b^2\alpha^2}\]
\[=\sqrt{a^2(\alpha^2+b^2)}\]
\[ = |\alpha||z|\]
Hence multiplying $z\in\mathbb{C}$ by a scalar $\alpha$, scales the magnitude of $z$.\\
Cases:\\
$\alpha > 1\quad$ extends the line straight along its original path (magnitude increases)\\
e.g. $z=2.5+2i$ and $\alpha = 2$
\[2z=5+4i\]
$0<\alpha<1\quad$ shrinks the line straight along its original path (magnitude decreases)\\
e.g. $z=2.5+2i$ and $\alpha = \frac{1}{2}$
\[\frac{1}{2}z=1.25+1i\]
$-1 <\alpha < 0\quad$ reverses direction and shrinks the line (rotation by 180 and shrinking in)
e.g. $z=2+4i$ and $\alpha = \frac{1}{2}$
\[-\frac{1}{2}z=-1-2i\]
$\alpha < -1\quad$ reverse direction and expands (rotation by 180 and magnitude increases)\\
e.g. $z=2.5+2i$ and $\alpha = -1$
\[-1z=2.5+2i\]
\end{framed}

\begin{framed}
    \noindent\textbf{DEF} $|$ Rotations by $\tau$ rad.\\
    Recall $z=re^{i\theta}$\\
    To obtain a rotated point/vector,
    \[w=re^{i(\theta+\tau)}\]
    To scale out by a factor of 2 and rotate by $\frac{\pi}{3} rad.$
    \[w=2re^{i(\pi+\frac{\pi}{3})}\]
\end{framed}
\noindent\textbf{Example 4.} What is the result of scaling $z=1+i$ by $\alpha=3$ and rotating $\frac{pi}{2}$ rad?\\
\[z=re^{i\theta}\]
\[r=\sqrt{1^2+1^2} = \sqrt{2}\]
\[\theta = tan^{-1}(1) = \frac{\pi}{4}\]
\[z=\sqrt{2}e^{i\frac{\pi}{4}}\]
\[w=3\sqrt{2}e^{i(\frac{pi}{4}) + \frac{\pi}{2}}\]
\[= 3\sqrt{2}e^{\frac{3\pi}{4}}\]
\[=3\sqrt{2}(cos(\frac{3\pi}{4})+sin(\frac{3\pi}{4}))\]
\[=-3+3i\]

\end{document}