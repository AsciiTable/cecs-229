\documentclass [12pt]{article}
\setlength{\oddsidemargin}{0.1in}
\setlength{\evensidemargin}{0.1in}
\setlength{\topmargin}{-.7in}
\setlength{\textheight}{9.25in}
\setlength{\textwidth}{6.5in}
\usepackage{caption}
\usepackage{subcaption}
\usepackage{enumerate}
\usepackage{framed}
\usepackage{epsfig}
\usepackage{changebar}
\usepackage{amsfonts}
\usepackage{amsmath}
\usepackage{amssymb}
\usepackage{graphicx}
\usepackage{amssymb}
\usepackage{stmaryrd}
\graphicspath{ {images/} }
\usepackage{listings}
\usepackage[usenames,dvipsnames]{color}
\usepackage{multicol}
\usepackage{mathtools}
\DeclarePairedDelimiter\ceil{\lceil}{\rceil}
\DeclarePairedDelimiter\floor{\lfloor}{\rfloor}
\setlength{\columnsep}{1cm}
% mathematical commands
\newcommand{\zos}{{\{ 0,1\}^{\ast}}}
\newcommand{\zoi}{{\{ 0,1\}^{\infty}}}
\newcommand{\zon}{{\{ 0,1\}}}
\newcommand{\zov}[1]{{\{ 0,1\}^{#1}}}
\newcommand{\ccc}{{{\cal C}}}
\newcommand{\gggg}{{{\cal G}}}
\newcommand{\nat}{{{\cal N}}}
\newcommand{\rr}{{{\bf RAND}}}
\newcommand{\pref}{{\sqsubset}}
\newcommand{\da}{{\downarrow}}
\newcommand{\ot}{{\otimes}}
\newcommand{\fann}{{\forall n\in \nat}}
\newcommand{\pow}{{{\cal P}}}
\newcommand{\nll}{{{\bf NULL}}}
\newcommand{\nvc}[1]{{{\bf e_{#1}}}}
\newcommand{\st}{{\Sigma_{2}^{A}}}
\newcommand{\ov}[1]{{\overline{#1}}}
\newcommand{\provided}{{\hspace{.1in}:-\hspace{.1in}}}
\begin{document}
\begin{center}\title*{\Large \S \; 4.1 Divisibility \& Modular Arithmetic}\\\author*{Jessica Wei} \end{center}
\normalsize
\noindent
% ----------INTRO
\textbf{\subsection*{Introduction}}
One of the goals of this course is to equip you with the mathematical tools necessary to understand
encryption methodologies and systems. The aim of this section of to introduce you to the underlying,
basic math of some of the most widely used cryptosystems.
\vspace{0.25in}
\vspace{0.25in}
\noindent
% NOTATION GLOSSARY IS HERE !!!!!!!!!!!!!!!!!!!!!!!!!!!!!!!!!!!!!!!!!!!!!
\textbf{\subsection*{Notations}}
\begin{framed}
\textbf{SYMBOL REFERENCE}\\
\\
${\rm I\!R}$ = set of real numbers and any number that is not complex.\\
$\mathbb{Z}$ = set of integers.\\
$\mathbb{Z^-}$ = set of negative integers.\\
$\mathbb{Z^+}$ = set of positive integers.\\
${\in}$ = ``in''\\
${\exists}$ = ``there exists''\\
${\exists!}$ = ``there exists a unique''
\end{framed}
\vspace{0.25in}
\textbf{\subsection*{Divisibility}}
% ----------DEF: argument, premises
\begin{framed}
\textbf{DEF} $|$ DIVIDES\\
\\
Let $a, b, \in \mathbb{Z}$ with $a \neq 0$. We say that $a$ divides $b$, denoted $a\;|\;b$, if $\exists
k \in \mathbb{Z}$ such that $b = a \cdot k$. In such a case, we also express this as $b \div a \in \mathbb{Z}$
\end{framed}
%*****EXAMPLE 1
\raggedright
\textbf{Example 1.} Determine whether each of the following statements are true.
\vspace{0.1in}
\begin{enumerate}[(a)]
\item $3\;| \;6$ \\
$6 = 3 \cdot k$ \quad $\Rightarrow$ \quad$k = 2 \in \mathbb{Z}$\\
\textbf{Answer:} True\\
\vspace{0.1in}
\item $6\;| \;3$ \\
$3 = 6 \cdot k$ \quad $\Rightarrow$ \quad$k = \frac{1}{2} \notin \mathbb{Z}$\\
\textbf{Answer:} False\\
\vspace{0.1in}
\item $3 \not | \; 5$\\
$5 = 3 \cdot k$ \quad $\Rightarrow$ \quad$k = \frac{5}{3} \notin \mathbb{Z}$\\
\textbf{Answer:} True\\
\end{enumerate}
%*****EXAMPLE 2
\raggedright
\textbf{Example 2} Let $n, d \in \mathbb{Z}^{+}$. How many positive integers not exceeding $n$ are divisible by $d$?\\
\quad $Consider:$ $n = 21, d = 2$. How many integers in the range of $1 \to 9$ are divisible by $2$?\\
\quad \quad(2k = multiple of 2)\\
\quad \quad $\Rightarrow$ \quad $2k \leq 21$\quad  $\Rightarrow$ \quad $k \leq 10$\\
\quad \quad $\therefore$ k = 2, 4, 6, 8, 10, 12, 14, 16, 18, 20\vspace{0.15in}\\
\quad  In general, $kd$ $\leq$ n \quad $\Rightarrow$\quad k $\leq$ $\floor{\frac{n}{d}}$\vspace{0.15in}\\
\quad \textbf{Answer: }$\therefore$ There are $\floor{\frac{n}{d}}$ positive integers not exceeding $n$ that are divisible by $d$.
\vspace{0.25in}

%*****THM: 4.1.1
\begin{framed}
\textbf{THM 4.1.1} $|$ \\
Let $a, b, c \in \mathbb{Z}$ and $c\neq 0$. \\
\begin{enumerate}[(i)]
\item If $a \; | \; b$ and $a\;|\;c$, then $a\;|\;(b + c)$.
\item If $a \; | \; b$, then $a\;|\;bc$ $\forall c \in \mathbb{Z}$.
\item If $a \;|\; b$ and $b\;|\;c$, then $a\;|\;c$.
\end{enumerate}
\end{framed}
\raggedright
\textbf{PROOF:} \\
%*************PROOF 1
\quad(i) $a + b$ by definition, $b = a \cdot k$ where $k\in\mathbb{Z}$. Similarly a $|$ c, then $c = a \cdot q$ where $q\in\mathbb{Z}$.\\
\quad\quad Hence, $b + c = a\cdot k + a \cdot q$\\
\quad\quad $\Rightarrow b + c = a (k + q)$ where $(k + q)\in\mathbb{Z}$\\
\quad\quad $\Rightarrow a | (b + c)$ \checkmark\\
\vspace{0.10in}
\quad\quad Consider $3|9$, $3|12$, then $3|(9+12) \Rightarrow 3|21$\\
\quad\quad\quad $12 + 9 = 3\cdot3 + 3\cdot4 = 3(3+4) = 3\cdot7$\\
\quad\quad\quad\textbf{Answer:} $3\cdot7$\\
\vspace{0.20in}
%*************PROOF 2
\quad (ii) If $a|b$, then by definition $b=a\cdot k$, where $k\in\mathbb{Z}$\\
\quad\quad \textbf{GOAL:} $b\cdot c = a\cdot integer$\\
\quad\quad From definition, $b\cdot c = a\cdot k\cdot c$ $\Rightarrow b\cdot c = a\cdot u$ where $u = k\cdot c \in \mathbb{Z}$\\
\quad\quad$\Rightarrow a | b\cdot c$\\
\vspace{0.20in}
%*************PROOF 3
\quad(iii) By assumption\\
\quad\quad $b = a\cdot k$ where $k\in\mathbb{Z}$\\
\quad\quad $c = b\cdot q$ where $q\in\mathbb{Z}$\\
\quad\quad $\Rightarrow c = a\cdot u$ where $u = k\cdot q \in\mathbb{Z}\Rightarrow a|c$
\vspace{0.25in}
\pagebreak

%*****THM: 4.1.2
\begin{framed}
\textbf{THM 4.1.2} $|$ \textbf{Division Algorithm }\\
\vspace{0.5cm}
Let $a\in \mathbb{Z}, d \in \mathbb{Z}^{+}$. Then, $\exists! \; q, r \in \mathbb{Z}$ satisfying $0 \leq r < d$ such that, \\
\[ a = d\cdot q +r\]
i.e. Any integer divided by positive integer $d$ will result in quotient $q$ and remainder $r$.
\end{framed}
\vspace{0.10in}
\textbf{PROOF:}\\
\quad\textbf{Case 1:} $d|a$
\quad\quad if $d|a$ then $\exists q \in \mathbb{Z}$ such that:\\
\quad\quad$a = d\cdot q \Rightarrow ad = d\cdot q + 0$\\
\quad\quad$\Rightarrow a = d \cdot q + r$ where $r = 0$ \checkmark\\
\vspace{0.1in}
\quad\textbf{Case 2:} $d\nmid a$\\
\quad\quad If $d\nmid a$, then mod can subtract an integer in the range $(0,d)$ such that the result is\\
\quad\quad divisible by $d$. (i.e. d = 4, a = 10)\\
\quad\quad **Note we are making the assumption that $d<a$.\\
\quad\quad Let such a number be $r$. Then:\\
\quad\quad\quad $d|a - r \Rightarrow a - r = d \cdot q$, where $q\in\mathbb{Z}$\\
\quad\quad\quad $\Rightarrow n=d\cdot q + r$ \checkmark\\
\vspace{0.25in}

%*****DEF: mod
\begin{framed}
\textbf{DEF} $|$ mod \\
Let $a, q, r \in \mathbb{Z}$ and $d \in \mathbb{Z}^{+}$ such that $a = d\cdot q + r$. We define, \\
\[a \text{ mod } d = r\]
whenever $a$ divided by $d$ results in remainder $r$.\\
i.e. $a \mod d$ is the remainder of $a \div d$
\end{framed}

%*****EXAMPLE 3
\raggedright
\textbf{Example 3.} Which of the following are true?
\begin{enumerate}[(a)]
\item $101 \mod 11 = 2$\\
\quad $101 = 11\cdot 9 + 2$\\
\quad \textbf{Answer:} True
\item $101 \mod 2 = 11$\\
\quad $101 = 2\cdot 50 + 1$\\
\quad \textbf{Answer:} False
\item $11 \mod 2 = 101$\\
\quad $11 = 2\cdot 5 + 1$\\
\quad \textbf{Answer:} False
\item $101 \mod 2 = 1$\\
\quad $101 = 2\cdot 50 + 1$\\
\quad \textbf{Answer:} True
\end{enumerate}
\vspace{0.1in}
%*****EXAMPLE 4
\raggedright
\textbf{Example 4.} What are the quotient and remainder when $-11$ is divided by 3?\\
\quad $-11 = 3(-4) + 1$ where $3(-4) = q$ and $1 = r$\\
\quad\textbf{Answer:} q = quotient = -4; r = remainder = 1
\vspace{0.25in}

%*****DEF: div
\begin{framed}
\textbf{DEF} $|$ div \\
\vspace{0.25cm}
If $a = d \cdot q + r $, then we write \[q = a \text{ div } d\] to express that $q$ is the quotient of $a \div
d$. \\
i.e. if $a$ div $d$, it returns the quotient of $a \div d$.
\end{framed}
\vspace{0.25cm}
%*****EXAMPLE 5
\raggedright
\textbf{Example 5.} Evaluate each of the following:
\begin{enumerate}[(a)]
\item $101 \text{ div } 11 = $\\
\quad $101 = 9 \cdot 11 + 2$\\
\quad\textbf{Answer:} 9
\item $-11 \text{ div } 3 = $\\
\quad $-11 = 3( -4) + 1$\\
\quad\textbf{Answer:} -4
\end{enumerate}
\vspace{0.1cm}

\subsubsection*{Modular Arithmetic}
%*****DEF: Congruent to b modulo m
\begin{framed}
\textbf{DEF} $|$ Congruent to $b$ modulo $m$\\
\vspace{0.5cm}
if $m|a-b$ where $a,b,m\in\mathbb{Z}$
We say $a$ is congruent to $b$ modulo $m$, denoted,
\[a \equiv b \text{ (mod }m)\]
if $\exists k \in \mathbb{Z}$ such that $a - b = m \cdot k$, i.e. $m\;|\; a - b$.\\
\end{framed}
%*****EXAMPLE 6
\raggedright
\textbf{Example 6.} Determine which of the following are true:
\begin{multicols}{2}
\begin{enumerate}[(a)]
\item $7 \equiv 3 \text{ (mod }4)$\\
\quad $4|(7-3) = 4|4$\\
\quad\textbf{Answer:} True
\item $7 \equiv 4 \text{ (mod }3)$\\
\quad $3|(7-4) = 3|3$\\
\quad\textbf{Answer:} True
\item $15 \equiv 3 \text{ (mod }5)$\\
\quad $5|(15-3) = 5|12$\\
\quad\textbf{Answer:} False
\item $15 \equiv 5 \text{ (mod }2)$\\
\quad $2|(15-5) = 2|10$\\
\quad\textbf{Answer:} True
\end{enumerate}
\end{multicols}
\pagebreak

%*****THM: 4.1.3
\begin{framed}
\textbf{THM 4.1.3} $|$ \\
\vspace{0.5cm}
Let $a, b \in \mathbb{Z}, m \in \mathbb{Z}^{+}$. Then, \\
\[a \equiv b \text{ (mod }m) \]
if and only if
\[a \text{ mod } m = b \text{ mod } m\]\\
i.e. if dividing $a$ by $m$ produces the same remainder as dividing $b$ by $m$, then $a$ will be congruent to $b$ modulo $m$ and vice versa because it's (if and only if) bi-conditional.
\end{framed}
\raggedright
\textbf{PROOF:}\\
\quad $(\shortrightarrow )$ Assume that $a \equiv b (\mod m)$. By the division algorithm, dividing $a$ by $m$ will result \\
\quad in $a = m\cdot q + r$ where $q, r \in \mathbb{Z}$. Similarly, dividing $b$ by $m$ will result in $b = m\cdot k + s$\\
\quad where $k, s \in \mathbb{Z}$. \\
\quad Hence, $a-b = (m\cdot q + r) - (m\cdot k + s) = (m\cdot q - m\cdot k) + (r - s)$\\
\quad (*)  $a-b = m(q-k) + (r - s)$\\
\quad Since $a\equiv b (\mod m)$, $m$ divides $a-b$, so the remainder portion of (*) must be 0.\\
\quad\quad i.e. $r-s = 0 \Rightarrow r=s$\\
\quad\quad $\Rightarrow a\mod m = b\mod m$\\
\vspace{0.3cm}
\quad $(\shortleftarrow)$ Now suppose $a\mod m \equiv b \mod m$.\\
\quad By the division algorithm:\\
\quad $a=m\cdot q + 3$, $b = m\cdot k + r$ \quad (notice remainder are the same the same by assumption)\\
\quad $\Rightarrow a-b = (m\cdot q + r)-(m\cdot k + r)$\\
\quad$\Rightarrow m\cdot q - m \cdot k + r - r$\\
\quad$\Rightarrow a - b = m\cdot q - m \cdot k = m(q-k)$\\
\quad$\Rightarrow a-b = m\cdot u $ where $u = q - k \in\mathbb{Z}$\\
\quad $\Rightarrow a \equiv b(\mod{m})$\\
\pagebreak

%*****THM: 4.1.4
\begin{framed}
\textbf{THM 4.1.4} $|$ \\
\vspace{0.5cm}
Let $m \in \mathbb{Z}^{+}$. If, $a \equiv b \text{ (mod }m) $ and $c \equiv d \text{ (mod }m) $, then\\
\begin{enumerate}[(i)]
\item $a+c \equiv b+d \text{ (mod } m)$
\item $ac \equiv bd \text{ (mod } m)$
\end{enumerate}
\end{framed}
\vspace{0.25cm}
\textbf{PROOF:}\\
\quad $a \equiv b (\mod m)$ and $c \equiv d (\mod m)$\\
\quad $a \equiv b (\mod m) \Rightarrow m | (a-b)$\\
\quad $c \equiv d (\mod m) \Rightarrow m | (c-d)$\\
\quad $\Rightarrow m|((a - b) + (c - d))$\\
\quad $\Rightarrow m|(a - b + c - d)$\\
\quad $\Rightarrow m|((a + c) + (-b - d))$\\
\quad $\Rightarrow m|((a + c) - (b + d))$\\
\quad $\Rightarrow a + c \equiv b + d (\mod m)$ \checkmark\\
\vspace{0.2in}
i.e. Given $15 \equiv 5 (\mod 2)$ is true and $7 \equiv 5 (\mod 3)$ is true,\\
\quad $\Rightarrow15 + 7 \equiv 5 + 5 (\mod 2) \Rightarrow 22 \equiv 10(\mod 2)$\\
\quad $\Rightarrow15\cdot 7 \equiv 5\cdot 5 (\mod 2)$\\
\quad $\Rightarrow 105\equiv 25 (\mod 2)$ \checkmark
\pagebreak

%*****COROLLARY: 4.1.4
\begin{framed}
\textbf{COROLLARY 4.1.4} $|$ \\
\vspace{0.5cm}
Let $m \in \mathbb{Z}^{+}$. \\
\begin{enumerate}[(i)]
\item $(a+b)\mod m = [(a\mod m ) + (b\mod m)]\mod m$
\item $(ab)\mod m = [(a\mod m ) \cdot (b\mod m)]\mod m$
\end{enumerate}
i.e. $33 + 5 \mod 2$\\
w/o: $33 + 5 \mod 2 = 32 mod 2 = 0$\\
w/: $[(33 \mod 2)  + (5 \mod 2)]\mod 2 = [1 + 1] \mod 2 = 2 \mod 2 = 0$
\end{framed}
\vspace{0.5cm}
\textbf{PROOF: TFYOG}\\
\vspace{0.25in}
%*****EXAMPLE 7
\raggedright
\textbf{Example 7.} Find each of the following without the aid of a calculator.
\begin{multicols}{2}
\begin{enumerate}[(a)]
\item $(3^4\mod 17)^{2}\mod 11$ \\
Let $B=34\mod 17$\\
Simplify $B = 3^4 \mod17$\\
\quad $B = 3^3 \cdot 3\mod 17$\\
\quad $= (3^3 \mod 17)(3\mod 17)\mod 17$\\
\quad $= 10\cdot 3 \mod 17$\\
\quad $= 30 \mod 17$\\
\quad\quad $\Rightarrow 30 = 1 \cdot 17 + 13$\\
$\therefore B = 13$\\
Goal: $B^2 \mod 11$\\
$= 13^2 \mod 11$\\
$= (13 \mod 11)(13 \mod 11)\mod 11$\\
$=2\cdot 2 \mod 11$\\
$=4 \mod 11 = 4$\\
\textbf{Answer: } 4

\columnbreak
\item $(99^2\mod 32)^3\mod 15$\\
Let $B=99^2 \mod 32$\\
Simplify $B$\\
\quad $B=(99 \mod32)(99\mod32)\mod32$\\
\quad $=3\mod32$\\
\quad $\therefore B=3$\\
$Goal: B^3\mod 15$\\
\quad $= 3^3 \mod 15$\\
\quad $= 27 \mod 15$\\
\quad \textbf{Answer:} 12
\end{enumerate}
\end{multicols}
\vspace{5in}
%*****DEF: Z_m
\begin{framed}
\textbf{DEF} $|$ Integers modulo $m$, $\mathbb{Z}_m$\\
\vspace{0.5cm}
The integers modulo $m$, denoted $\mathbb{Z}_m$, are defined as
\[\mathbb{Z}_m = \{0,1, 2,3, \dots, m-1 \}\]
\textbf{Operations on $\mathbb{Z}_m$:}
\begin{enumerate}[]
\item $a +_{m} b = (a+b)\mod m$
\item $a \; \cdot_{m} b = (ab)\mod m$
\end{enumerate}
\end{framed}
%*****EXAMPLE 8
\raggedright
\textbf{Example 8.} Find the following:
\begin{multicols}{2}
\begin{enumerate}[(a)]
\item $7 +_{11} 9 =$\\
\quad $=7+9\mod11$\\
\quad $=16\mod11$\\
\quad \textbf{Answer:} 5
\item $7 \cdot_{11} 9 =$\\
\quad $=7\cdot9\mod11$\\
\quad $=63\mod11$\\
\quad \textbf{Answer:} 8
\end{enumerate}
\end{multicols}
\end{document}
\end{document}
