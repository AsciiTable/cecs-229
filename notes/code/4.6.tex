\documentclass [12pt]{article}
\setlength{\oddsidemargin}{0.1in}
\setlength{\evensidemargin}{0.1in}
\setlength{\topmargin}{-.7in}
\setlength{\textheight}{9.25in}
\setlength{\textwidth}{6.5in}
\usepackage{caption}
\usepackage{subcaption}
\usepackage{enumerate}
\usepackage{framed}
\usepackage{epsfig}
\usepackage{changebar}
\usepackage{amsfonts}
\usepackage{amsmath}
\usepackage{amssymb}
\usepackage{graphicx}
\usepackage{amssymb}
\usepackage{stmaryrd}
\graphicspath{ {images/} }
\usepackage{listings}
\usepackage[usenames,dvipsnames]{color}
\usepackage{multicol}
\usepackage{mathtools}
\DeclarePairedDelimiter\ceil{\lceil}{\rceil}
\DeclarePairedDelimiter\floor{\lfloor}{\rfloor}
\setlength{\columnsep}{1cm}
% mathematical commands
\newcommand{\zos}{{\{ 0,1\}^{\ast}}}
\newcommand{\zoi}{{\{ 0,1\}^{\infty}}}
\newcommand{\zon}{{\{ 0,1\}}}
\newcommand{\zov}[1]{{\{ 0,1\}^{#1}}}
\newcommand{\ccc}{{{\cal C}}}
\newcommand{\gggg}{{{\cal G}}}
\newcommand{\nat}{{{\cal N}}}
\newcommand{\rr}{{{\bf RAND}}}
\newcommand{\pref}{{\sqsubset}}
\newcommand{\da}{{\downarrow}}
\newcommand{\ot}{{\otimes}}
\newcommand{\fann}{{\forall n\in \nat}}
\newcommand{\pow}{{{\cal P}}}
\newcommand{\nll}{{{\bf NULL}}}
\newcommand{\nvc}[1]{{{\bf e_{#1}}}}
\newcommand{\st}{{\Sigma_{2}^{A}}}
\newcommand{\ov}[1]{{\overline{#1}}}
\newcommand{\provided}{{\hspace{.1in}:-\hspace{.1in}}}
\begin{document}
\begin{center}\title*{\Large \S \; 4.6 Encryption}\\\author*{Jessica Wei} \end{center}
\normalsize
\noindent
%-------------Encryption
\textbf{\subsection*{Encryption}}
\begin{framed}
\noindent\textbf{DEF} $|$ Encryption\\
The process by which a message is made secret.
\end{framed}
\textbf{\subsection*{Classic Cryptography}}
\noindent\textbf{I. Shift and Affine Ciphers}\\
Process
\begin{enumerate}[\quad1.]
    \item Assign a numeric value to each letter\\
    \quad ${A,B,...,Z} \Rightarrow {00,01,...25}$
    \item Apply a shift $k$ to the value that only the intended recipient knows about
\end{enumerate}
\textbf{Example 1.} Julius Cesar
\begin{enumerate}[\quad]
    \item Encrypted messages by shifting each letter three letters over. Use this shift to encrypt
    \item "MEET YOU IN THE PARK"
\end{enumerate}
\begin{enumerate}[\quad\quad1.]
    \item M E E T Y O U  I N  T H E  P A R K\\
    12 4 4 19 24 14 20 8 13 19 7 4 15 0 17 10
    \item 15 7 7 22 1 17 23 11 16 22 10 7 18 3 20 13\\
    \textbf{Answer: }PHHW BRX LQ RXL QWKHSDUN
\end{enumerate}
For this particular example, we can express an encryption function that describes the rule applied to each letter: 
\[f(x)=p+3\mod26\]
where $p\in{00,01,...,25}$\\
NOTE: This cipher is not very secure because you can break it by:
\begin{enumerate}[i)]
    \item Brute Force: test every one of the 26 possible shifts
    \item checking letter frequencies w/ popular letters
\end{enumerate}
\vspace{0.2in}

\noindent\textbf{*Encryption Function:}\\
We can further generalize the encryption function to be of the form:
\[f(p) = ap+b\mod 26\]
where $a,b\in\mathbb{Z}$ and gcd($a$, 26) = 1.\\
This last condition ensures that the function is bijective. (i.e. that there will be an inverse function to decrypt)\\
\begin{enumerate}[*]
    \item If $a=1$, $f(p)=p+b\mod26$, then we have a shift cipher with key $b$
    \item If $a>1$, then we call the function an affine transformation.
\end{enumerate}
The key is necessary for easy decryption.
\vspace{0.2in}

\noindent\textbf{*Decryption Function:}\\
\noindent\textbf{Case:} Shift Cipher\\
\[f^{-1}(q)=q-b\mod26\]
\noindent\textbf{Case:} Affine Transformation\\
\[q=ap+b\mod26 \Rightarrow ap+b=q\mod26\]
\[ap\equiv q-b\mod26\]
Since gcd($a$, 26) = 1, $\exists \Bar{a}\in\mathbb{Z}$
\[\Bar{a}ap\equiv \Bar{a}(q-b)\mod26\]
\[\Rightarrow p \equiv \Bar{a}(q-b)\mod26\]
$\therefore$ Decryption Function = $f^{-1}(q)=\Bar{a}(q-b)\mod26$\\
\vspace{0.2in}
\\
\noindent\textbf{Example 2.} An Affine transformation was applied with $a=7$ and $b=3$ to encode a letter. The encrypted letter is V. What was the original letter? *Note V = 21
\[f^{-1}(q) = \Bar{a}(q-3)\mod26\]
Need: Inverse of $7\mod26$
\begin{enumerate}[\quad]
    \item $26=3\cdot7+5\Rightarrow5=26-3\cdot7$
    \item $7=1\cdot5+2\Rightarrow2=7-5$
    \item $5=2\cdot2+1\Rightarrow1=5-2\cdot2$
    \item $2=2\cdot1+0$
    \item ------
    \item $1=5-2(7-5)=3\cdot5-2\cdot7$
    \item $1=3(26-3\cdot7)-2\cdot7=3\cdot26-11\cdot7$
    \item inverse = -11 = $15\mod26\Rightarrow\Bar{a}=15$
    \item ------
    \item $f^{-1}(q) = 15(q-3)\mod26$
    \item $f^{-1}(21) = 15(21-3)\mod26$
    \item $=15\cdot18\mod26=45\cdot6\mod26$
    \item $=(45\mod26)(6\mod26)=114\mod26=10$
    \item\textbf{Answer:} $k$
\end{enumerate}
\vspace{0.2in}

\noindent\textbf{II. Transportation Ciphers}
Process
\begin{enumerate}[1.]
    \item Divide the string of letters into blocks of a given size and add a padding in the last block if necessary (xx)
    \item Permute the letters of each block
\end{enumerate}
\noindent\textbf{Example 3.} Use the permutation $\sigma:{1,2,3,4}$ defined by $\sigma(1)=3$, $\sigma(2)=1$, $\sigma(3)=4$, $\sigma(4)=2$ to encrypt the message "PIRATE ATTACK".
\begin{enumerate}[\quad1.]
    \item PIRA TEAT TACK
    \item IAPR ETTA AKTC
\end{enumerate}
\quad\textbf{Answer: }IAPRETTAAKTC
\vspace{0.2in}
\\
\noindent\textbf{Example 4.} Decrypt the message SWUE TRAE OEHS if the encryption that was used $\sigma(1)=3$, $\sigma(2)=1$, $\sigma(3)=4$, $\sigma(4)=2$
\begin{enumerate}[\quad]
    \item $\sigma^{-1}(1)=2$, $\sigma^{-1}(2)=4$, $\sigma^{-1}(3)=4$, $\sigma^{-1}(4)=3$
    \item USEW ATER HOSE
    \item\textbf{Answer:} Use Water Hose
\end{enumerate}
\begin{framed}
\noindent\textbf{DEF} $|$ Cryptosystem\\
A cryptosystem is a 5-tuple $(P,C,K,E,D)$ where
\begin{enumerate}[\quad]
    \item P: set of plaintext strings
    \item C: set of ciphertext strings
    \item K: set of possible keys
    \item E: set of encryption functions
    \item D: set of decryption functions
\end{enumerate}
Given some key $k \in K$
\begin{enumerate}[$\cdot$]
    \item $E_k\in E$ is an encryption function with key $K$
    \item $D_k\in D$ is a decryption function with key $K$
\end{enumerate}
\end{framed}
\pagebreak
\noindent\textbf{Example 5.} Describe the family of shift ciphers as a cryptosystem.
\begin{enumerate}[\quad]
    \item $C=K=P={00,01,...,25}=\mathbb{Z}_{26}$ 
    \item $E={p+b\mod26|q,b\in\mathbb{Z}_{26}}$
    \item $D={q-b\mod26|q,b\in\mathbb{Z}_{26}}$
\end{enumerate}
\textbf{\subsection*{RSA}}
\begin{framed}
\noindent\textbf{DEF} $|$ Private Cryptosystem\\
A cryptosystem where knowing a key allows you to find the decryption function.
\end{framed}
\begin{framed}
\noindent\textbf{DEF} $|$ Public Cryptosystem\\
A cryptosystem where knowing a key does not reveal the decryption function.
\end{framed}
\noindent\textbf{Key:} $(n,e)$ where $n,e\in\mathbb{Z}$
\begin{enumerate}[\quad(i)]
    \item $n=p\cdot q$ where $p,q\in\mathbb{Z}$\quad primes(200+ digits)
    \item $e$ satisfies gcd($e$, $(p-1)(q-1)$) = 1
\end{enumerate}
\noindent\textbf{Encryption Function:}
\[c=m^e\mod n\]
\noindent\textbf{Process:}
\begin{enumerate}[\quad1.]
    \item ${A,B,...Z}\Rightarrow {00,01,...25}$
    \item Concatenate the digits into blocks of even size (all blocks must be the same size)\\
    \quad *In practical applications, the block size depends on key size\\
    \quad *In our case, adopt the convention that size = \# of digits in 2525..25 where 2525..25$<n$\\
    \quad e.g. $n = 3821$ $\Rightarrow$ $2525<3821\Rightarrow5=4$
    \item Apply the encryption function to each block $m_i$
    \[c_i = m_i^e\mod n\]
    \item Concatenate the cipher block $c$
\end{enumerate}
\noindent\textbf{Example 6.} Encrypt "STOP", key: (2537,13)
\begin{enumerate}[\quad1.]
    \item STOP $\rightarrow$ 18191415
    \item $n=2547$, $2525<2537\Rightarrow$ size = 4\\
    \quad $m_1=1819$, $m_2=1415$
    \item $c_1=1819^{13}\mod2537=2081$\\
    \quad $c_2=1415^13\mod2537=2182$
    \item \textbf{Cipher:} 20812182
\end{enumerate}
\pagebreak
\noindent\textbf{Decryption}\\
\textbf{*Goal:} $c=m^e\mod n$ obtain $m\in$ plaintext digits\\
Information about key:
\begin{enumerate}[\quad(1)]
    \item $n=p\cdot q$ where $p,q$ are primes
    \item gcd($e$, $(p-1)(q-1)$) = 1
\end{enumerate}
Consider (2), we are able to find an inverse of $e,\Bar{e}_i\mod (p-1)(q-1)$ i.e.\\
\[e\Bar{e}-1=(p-1)(q-1)k\quad\quad k\in\mathbb{Z}\]
\[e\Bar{e}=(p-1)(q-1)k+1\]
\[c\equiv m^e\mod n\]
\[c^{\Bar{e}}\equiv (m^e)^{\Bar{e}}\mod n\]
\[c^{\Bar{e}}\equiv m^{e{\Bar{e}}}\mod n\]
\[c^{\Bar{e}}\equiv m^{(p-1)(q-1)k+1}\mod n\quad\quad m^{a+b}\equiv m^a\cdot m^b\]
\[c^{\Bar{e}}\equiv m^{(p-1)(q-1)k}\mod n\]
\[\Rightarrow c^{\Bar{e}}-m^{(p-1)(q-1)k}\cdot m = ny\quad\quad y\in\mathbb{Z}\]
\[\Rightarrow c^{\Bar{e}}=ny+m^{(p-1)(q-1)k}\cdot m\]
\[\Rightarrow c^{\Bar{e}}=pqy+m^{(p-1)(q-1)k}\cdot m\]
\[\Rightarrow c^{\Bar{e}}=m^{(p-1)(q-1)k}\cdot m\mod p\]
\[\Rightarrow c^{\Bar{e}}=m^{(p-1)(q-1)k}\cdot m\mod q\]
\vspace{0.2in}


\noindent Recall FLT: If $p$ is prime \& $p\nmid a$ then $a^{p-1}\equiv 1\mod p$\\
\noindent\textbf{Case 1.} $p\nmid m$ and $q\nmid m$\\
\quad $m^{p-1}\equiv1\mod p$ and $m^{q-1}\equiv1\mod q$ (5)\\
Hence by (5), we obtain the system
\[c^{\Bar{e}}=(m^{(p-1)})^{(q-1)k}\cdot m\mod p = 1^{(q-1)k}\cdot ,\mod p\equiv m\mod p\]
\[\Rightarrow c^{\Bar{e}}\equiv m\mod p\]
\[*c^{\Bar{e}}\equiv(m^{(q-1)})^{(p-1)k}\cdot m \mod q \equiv 1^{(q-1)k}\cdot m\mod q\]
\[c^{\Bar{e}}\equiv m\mod q\]
\[c^{\Bar{e}}\equiv m\mod p\]
\[c^{\Bar{e}}\equiv m\mod q\]
Solve this System of Congruence to obtain plaintext block $m$
\pagebreak
Solving by substitution results in the substitution
\[m=c^{\Bar{e}}\mod n\]
\noindent\textbf{Case 2.} $p|m$ or $q|m$ (rarely ever happens)\\
Use CRT to show that 
\[c^{\Bar{e}\equiv m^{(p-1)(q-1)k}}\cdot m\mod p\]
\[c^{\Bar{e}\equiv m^{(p-1)(q-1)k}}\cdot m\mod q\]
still results in the same decryption function.\\
\vspace{0.2in}

\noindent\textbf{Summary of RSA}\\
*Encryption key ($n$, $e$) where $n = p\cdot q$ where $p,q$ are primes
*Encryption Function: $c=m \mod m$ where m = block of letters in digit form $25\cdot25\cdot25<n$ if digits = size of block\\
\vspace{0.1cm}

\noindent*Decryption Function \\
\quad $e=m^{\Bar{e}\mod n}$\\
\quad e = inverse of $e\mod(p-1)(q-1)$\\
\vspace{0.2in}

\noindent\textbf{Example 7.} Decrypt 09810461 with key(2537, 13)
\begin{enumerate}[\quad1.]
    \item $p=43$, $q=59$, $2537=43\cdot59$
    \item $\Bar{e}\cdot13\equiv1\mod42\cdot58$\\
    \quad $\Bar{e}\cdot13\equiv1\mod2436$\\
    \quad $937\cdot13-5\cdot2436=1\Rightarrow\Bar{e}=937$
    \item $2525 < 2537\rightarrow$ block size = 4
    \item $m_1 = 0981$, $m_2=0461$\\
    \quad $c_1=m_1^{\Bar{e}}\mod n\Rightarrow c_1 = 981^{937}\mod2537=704$\\
    \quad $c_2=m_2^{\Bar{e}}\mod n\Rightarrow c_2=461^{937}\mod2537=1115$
    \item 07041115 = HELP
\end{enumerate}
*Note: RSA is considered a public cryptosystem because even though $(n,e)$ is public, finding the prime factorization of $n$ is impossible within a reasonable amount of time where $n$ is very large.\\
RSA - 768 Case: $n$ had 232 digits and it took several computers 2 years to find the factorization. If 1 computer had done it alone, it wouldve have taken 2,000 years.
\end{document}