\documentclass [12pt]{article}
\setlength{\oddsidemargin}{0.1in}
\setlength{\evensidemargin}{0.1in}
\setlength{\topmargin}{-.7in}
\setlength{\textheight}{9.25in}
\setlength{\textwidth}{6.5in}
\usepackage{caption}
\usepackage{subcaption}
\usepackage{enumerate}
\usepackage{framed}
\usepackage{epsfig}
\usepackage{changebar}
\usepackage{amsfonts}
\usepackage{amsmath}
\usepackage{amssymb}
\usepackage{graphicx}
\usepackage{amssymb}
\usepackage{stmaryrd}
\graphicspath{ {images/} }
\usepackage{listings}
\usepackage[usenames,dvipsnames]{color}
\usepackage{multicol}
\usepackage{mathtools}
\DeclarePairedDelimiter\ceil{\lceil}{\rceil}
\DeclarePairedDelimiter\floor{\lfloor}{\rfloor}
\setlength{\columnsep}{1cm}
% mathematical commands
\newcommand{\zos}{{\{ 0,1\}^{\ast}}}
\newcommand{\zoi}{{\{ 0,1\}^{\infty}}}
\newcommand{\zon}{{\{ 0,1\}}}
\newcommand{\zov}[1]{{\{ 0,1\}^{#1}}}
\newcommand{\ccc}{{{\cal C}}}
\newcommand{\gggg}{{{\cal G}}}
\newcommand{\nat}{{{\cal N}}}
\newcommand{\rr}{{{\bf RAND}}}
\newcommand{\pref}{{\sqsubset}}
\newcommand{\da}{{\downarrow}}
\newcommand{\ot}{{\otimes}}
\newcommand{\fann}{{\forall n\in \nat}}
\newcommand{\pow}{{{\cal P}}}
\newcommand{\nll}{{{\bf NULL}}}
\newcommand{\nvc}[1]{{{\bf e_{#1}}}}
\newcommand{\st}{{\Sigma_{2}^{A}}}
\newcommand{\ov}[1]{{\overline{#1}}}
\newcommand{\provided}{{\hspace{.1in}:-\hspace{.1in}}}
\begin{document}
\begin{center}\title*{\Large \S \; 4.6 Encryption}\\\author*{Jessica Wei} \end{center}
\normalsize
\noindent
%-------------Encryption
\textbf{\subsection*{Encryption}}
\begin{framed}
\noindent\textbf{DEF} $|$ Encryption\\
The process by which a message is made secret.
\end{framed}
\textbf{\subsection*{Classic Cryptography}}
\noindent\textbf{I. Shift and Affine Ciphers}\\
Process
\begin{enumerate}[\quad1.]
    \item Assign a numeric value to each letter\\
    \quad ${A,B,...,Z} \Rightarrow {00,01,...25}$
    \item Apply a shift $k$ to the value that only the intended recipient knows about
\end{enumerate}
\textbf{Example 1.} Julius Cesar
\begin{enumerate}[\quad]
    \item Encrypted messages by shifting each letter three letters over. Use this shift to encrypt
    \item "MEET YOU IN THE PARK"
\end{enumerate}
\begin{enumerate}[\quad\quad1.]
    \item M E E T Y O U  I N  T H E  P A R K\\
    12 4 4 19 24 14 20 8 13 19 7 4 15 0 17 10
    \item 15 7 7 22 1 17 23 11 16 22 10 7 18 3 20 13\\
    \textbf{Answer: }PHHW BRX LQ RXL QWKHSDUN
\end{enumerate}
For this particular example, we can express an encryption function that describes the rule applied to each letter: 
\[f(x)=p+3\mod26\]
where $p\in{00,01,...,25}$\\
NOTE: This cipher is not very secure because you can break it by:
\begin{enumerate}[i)]
    \item Brute Force: test every one of the 26 possible shifts
    \item checking letter frequencies w/ popular letters
\end{enumerate}
\vspace{0.2in}
\\
\textbf{*Encryption Function:}
We can further generalize the encryption function to be of the form:
\[f(p) = ap+b\mod 26\]
where $a,b\in\mathbb{Z}$ and gcd($a$, 26) = 1.\\
This last condition ensures that the function is bijective. (i.e. that there will be an inverse function to decrypt)\\
\begin{enumerate}[*]
    \item If $a=1$, $f(p)=p+b\mod26$, then we have a shift cipher with key $b$
    \item If $a>1$, then we call the function an affine transformation.
\end{enumerate}
The key is necessary for easy decryption.
\vspace{0.2in}
\\
\textbf{*Decryption Function:}
\noindent\textbf{Case:} Shift Cipher\\
\[f^{-1}(q)=q-b\mod26\]
\noindent\textbf{Case:} Affine Transformation\\
\[q=ap+b\mod26 \Rightarrow ap+b=q\mod26\]
\[ap\equiv q-b\mod26\]
Since gcd($a$, 26) = 1, $\exists \Bar{a}\in\mathbb{Z}$
\[\Bar{a}ap\equiv \Bar{a}(q-b)\mod26\]
\[\Rightarrow p \equiv \Bar{a}(q-b)\mod26\]
$\therefore$ Decryption Function = $f^{-1}(q)=\Bar{a}(q-b)\mod26$\\
\vspace{0.2in}
\\
\noindent\textbf{Example 2.} An Affine transformation was applied with $a=7$ and $b=3$ to encode a letter. The encrypted letter is V. What was the original letter? *Note V = 21
\[f^{-1}(q) = \Bar{a}(q-3)\mod26\]
Need: Inverse of $7\mod26$
\begin{enumerate}[\quad]
    \item $26=3\cdot7+5\Rightarrow5=26-3\cdot7$
    \item $7=1\cdot5+2\Rightarrow2=7-5$
    \item $5=2\cdot2+1\Rightarrow1=5-2\cdot2$
    \item $2=2\cdot1+0$
    \item ------
    \item $1=5-2(7-5)=3\cdot5-2\cdot7$
    \item $1=3(26-3\cdot7)-2\cdot7=3\cdot26-11\cdot7$
    \item inverse = -11 = $15\mod26\Rightarrow\Bar{a}=15$
    \item ------
    \item $f^{-1}(q) = 15(q-3)\mod26$
    \item $f^{-1}(21) = 15(21-3)\mod26$
    \item $=15\cdot18\mod26=45\cdot6\mod26$
    \item $=(45\mod26)(6\mod26)=114\mod26=10$
    \item\textbf{Answer:} $k$
\end{enumerate}
\vspace{0.2in}
\\
\noindent\textbf{II. Transportation Ciphers}\\
Process
\begin{enumerate}[1.]
    \item Divide the string of letters into blocks of a given size and add a padding in the last block if necessary (xx)
    \item Permute the letters of each block
\end{enumerate}
\noindent\textbf{Example 3.} Use the permutation $\sigma:{1,2,3,4}$ defined by $\sigma(1)=3$, $\sigma(2)=1$, $\sigma(3)=4$, $\sigma(4)=2$ to encrypt the message "PIRATE ATTACK".
\begin{enumerate}[\quad1.]
    \item PIRA TEAT TACK
    \item IAPR ETTA AKTC
\end{enumerate}
\quad\textbf{Answer: }IAPRETTAAKTC
\vspace{0.2in}
\\
\noindent\textbf{Example 4.} Decrypt the message SWUE TRAE OEHS if the encryption that was used $\sigma(1)=3$, $\sigma(2)=1$, $\sigma(3)=4$, $\sigma(4)=2$
\begin{enumerate}[\quad]
    \item $\sigma^{-1}(1)=2$, $\sigma^{-1}(2)=4$, $\sigma^{-1}(3)=4$, $\sigma^{-1}(4)=3$
    \item USEW ATER HOSE
    \item\textbf{Answer:} Use Water Hose
\end{enumerate}
\begin{framed}
\noindent\textbf{DEF} $|$ Cryptosystem\\
A cryptosystem is a 5-tuple $(P,C,K,E,D)$ where
\begin{enumerate}[\quad]
    \item P: set of plaintext strings
    \item C: set of ciphertext strings
    \item K: set of possible keys
    \item E: set of encryption functions
    \item D: set of decryption functions
\end{enumerate}
Given some key $k \in K$
\begin{enumerate}[$\cdot$]
    \item $E_k\in E$ is an encryption function with key $K$
    \item $D_k\in D$ is a decryption function with key $K$
\end{enumerate}
\end{framed}
\pagebreak
\noindent\textbf{Example 5.} Describe the family of shift ciphers as a cryptosystem.
\begin{enumerate}[\quad]
    \item $C=K=P={00,01,...,25}=\mathbb{Z}_{26}$ 
    \item $E={p+b\mod26|q,b\in\mathbb{Z}_{26}}$
    \item $D={q-b\mod26|q,b\in\mathbb{Z}_{26}}$
\end{enumerate}
\end{document}